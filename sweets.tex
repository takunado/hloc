\documentclass{article}
%\documentclass[twoside,aps,prl,preprint,10pt,notitlepage]{revtex4-1}
\usepackage[active,tightpage]{preview}
%\usepackage[head=12pt,includehead,inner=4cm,outer=1cm,top=1cm,bottom=1.5cm]{geometry}
\usepackage[utf8]{inputenc}
\usepackage{chemfig}
\usepackage{longtable}
\usepackage{amssymb}
\usepackage{multirow}
\usepackage[verbose]{wrapfig}
\usepackage{textpos}
%\usepackage{fancyhdr}
\usepackage{makecell}
\usepackage{array}
\usepackage{amsmath}
\usepackage{ifthen}
\usepackage{cleveref}


%\usepackage{titlesec}

%\setlength{\TPHorizModule}{50cm}
%\setlength{\TPVertModule}{\textheight}
%\renewcommand{\PreviewBorder}{1in}
%\newcommand{\Newpage}{\end{preview}\begin{preview}}


%\pagestyle{fancy}
%\fancyheadoffset{0pt}
%\renewcommand{\sectionmark}[1]{\markboth{#1}{}}
%\fancyhf{}
%\fancyhead[LE,RO]{\leftmark}
%\fancyhead[RE,LO]{\thepage}
%\fancypagestyle{plain}{%
%  \fancyhead{} % get rid of headers
%  \renewcommand{\headrulewidth}{0pt} % and the line
%}

%\titlespacing\section{0pt}{15pt plus 4pt minus 2pt}{10pt plus 2pt minus 2pt}
%\titlespacing\subsection{0pt}{12pt plus 4pt minus 2pt}{8pt plus 2pt minus 2pt}
%\titlespacing\subsubsection{0pt}{8pt plus 4pt minus 2pt}{5pt plus 2pt minus 2pt}
%\setlength{\TPHorizModule}{\textwidth}
%\setlength{\TPVertModule}{\textheight}
\setlength{\parskip}{10mm}
%\let\stdsection\section
%\renewcommand\section{\newpage\stdsection}

\newcommand{\vpha}{\vphantom{A}}
\newcommand{\degree}{\ensuremath{^\circ}}
\newcommand{\chemar}[2][]{
  \schemestart[{},#1]
  \vpha\arrow{#2}\vpha
  \schemestop
}
\newcommand{\chemeq}[3][]{
  \quad\chemnameinit{}
  \noindent
  \begin{equation}
    \schemestart[][base]
    #3
    \schemestop
    \hskip \textwidth minus \textwidth
    \ifthenelse{\equal{#1}{}}{}{\cref{#1}\Leftarrow}
    \ifthenelse{\equal{#2}{}}{}{\label{#2}}
  \end{equation}
}
\newcommand{\precyc}{\vpha-[,0.1,,,draw=none]}
\newcommand{\grn}[1]{\vpha{\color{green}#1}}
\newcommand{\blu}[1]{\vpha{\color{blue}#1}}
\newcommand{\blk}[1]{\vpha{\color{black}#1}}
\newcommand{\red}[1]{\vpha{\color{red}#1}}
\newcommand{\brw}[1]{\vpha{\color{brown}#1}}
\newcommand{\ppl}[1]{\vpha{\color{purple}#1}}
\newcommand{\vlt}[1]{\vpha{\color{violet}#1}}
\definesubmol\O{\red{O}}
\definesubmol\N{\blu{N}}
\definesubmol\S{\brw{S}}
\newcommand{\X}{H}
\newcommand{\M}{\chemabove{\N}{\X}}
\newcommand{\K}{\chembelow{\N}{\X}}
\definesubmol\R{ \chemskipalign{\mbox{\scriptsize R}} }
\definesubmol{x}{(<[4]H)(<[0]\O|H)}
\definesubmol{y}{(<[0]H)(<[4]H|\O)}


\usetikzlibrary{decorations}
\pgfdeclaredecoration{ddbond}{initial}{
  \state{initial}[width=2pt]{
    \pgfpathlineto{\pgfpoint{4pt}{0pt}}
    \pgfpathmoveto{\pgfpoint{1pt}{2pt}}
    \pgfpathlineto{\pgfpoint{2pt}{2pt}}
    \pgfpathmoveto{\pgfpoint{4pt}{0pt}}
  }
  \state{final}{
    \pgfpathlineto{\pgfpointdecoratedpathlast}
  }
}
\tikzset{lddbond/.style={decorate, decoration=ddbond}}
\tikzset{rddbond/.style={decorate, decoration={ddbond, mirror}}}



















\begin{document}

\abovedisplayskip=9pt plus 3pt minus 9pt
\abovedisplayshortskip=0pt plus 3pt
\belowdisplayskip=9pt plus 3pt minus 9pt
\belowdisplayshortskip=4pt plus 3pt minus 4pt



%\setatomsep{2em}
\setchemfig{atom sep=2.2em}
%\setcrambond{2pt}{0.75pt}{0.75pt}
\setchemfig{cram width = 4pt}
\setchemfig{cram dash width = 0.75pt}
\setchemfig{cram dash sep = 0.75pt}
\setchemfig{arrow label sep = 1.5pt}
\setchemfig{bond join=true}
\setchemfig{atom style={scale=1.6}}
%\setchemfig{bond style={thick}}


\begin{preview}


  \chemfig{H|\O-[:120]-[:60](<:[:15]\O|H)(<[:-15]H)
                     -[:120](<:[:165]H)(<[:195]\O|H)
                      -[:60](<:[:15]H)(<[:-15]\O|H)
                     -[:120](<:[:165]H)(<[:195]\O|H)-[:60]=[0]\O}
\\
\\



\end{preview}

\end{document}



\chemname{\chemfig{[2]\O|H-[3]-!x-=[1]\O}}{glyceraldehyde}
\\   \\

\chemname{\chemfig{[2]\O|H-[3]-!x-!x-=[1]\O}}{erythrose}
\\   \\

\chemname{\chemfig{[2]\O|H-[3]-!x-!y-=[1]\O}}{threose}
\\   \\

\chemname{\chemfig{[2]\O|H-[3]-!x-!x-!x-=[1]\O}}{ribose}
\\   \\

\chemname{\chemfig{[2]\O|H-[3]-!x-!x-!y-=[1]\O}}{arabinose}
\\   \\

\chemname{\chemfig{[2]\O|H-[3]-!x-!y-!x-=[1]\O}}{xylose}
\\   \\

\chemname{\chemfig{[2]\O|H-[3]-!x-!y-!y-=[1]\O}}{lyxose}
\\   \\

\chemname{\chemfig{[2]\O|H-[3]-!x-!x-!x-!x-=[1]\O}}{allose}
\\   \\

\chemname{\chemfig{[2]\O|H-[3]-!x-!x-!x-!y-=[1]\O}}{altrose}
\\   \\

\chemname{\chemfig{[2]\O|H-[3]-!x-!x-!y-!x-=[1]\O}}{glucose}
\\   \\

\chemname{\chemfig{[2]\O|H-[3]-!x-!x-!y-!y-=[1]\O}}{mannose}
\\   \\

\chemname{\chemfig{[2]\O|H-[3]-!x-!y-!x-!x-=[1]\O}}{gulose}
\\   \\

\chemname{\chemfig{[2]\O|H-[3]-!x-!y-!x-!y-=[1]\O}}{idose}
\\   \\

\chemname{\chemfig{[2]\O|H-[3]-!x-!y-!y-!x-=[1]\O}}{galaclose}
\\   \\

\chemname{\chemfig{[2]\O|H-[3]-!x-!y-!y-!y-=[1]\O}}{talose}
\\   \\


\chemname{\chemfig{[2]\O|H-[3]-!x-(=[0]\O)(-[2]-[1]\O|H)}}{erythulose}
\\   \\

\chemname{\chemfig{[2]\O|H-[3]-!x-!y-(=[0]\O)(-[2]-[1]\O|H)}}{xylulose}
\\   \\

\chemname{\chemfig{[2]\O|H-[3]-!x-!x-(=[0]\O)(-[2]-[1]\O|H)}}{ribulose}
\\   \\

\chemname{\chemfig{[2]\O|H-[3]-!x-!y-!y-(=[0]\O)(-[2]-[1]\O|H)}}{tagatose}
\\   \\

\chemname{\chemfig{[2]\O|H-[3]-!x-!y-!x-(=[0]\O)(-[2]-[1]\O|H)}}{sorbose}
\\   \\

\chemname{\chemfig{[2]\O|H-[3]-!x-!x-!x-(=[0]\O)(-[2]-[1]\O|H)}}{psicose}
\\   \\

\chemname{\chemfig{[2]\O|H-[3]-!x-!x-!y-(=[0]\O)(-[2]-[1]\O|H)}}{fructose}
\\   \\

\chemname{\chemfig{[2]\O|H-[3]-!x-!x-!x-!y-(=[0]\O)(-[2]-[1]\O|H)}}{sedoheptulose}
\\   \\

%\end{tabular}

\chemfig{[2]\O|H-[3]-!x-!x-!y-!x-=[1]\O}

\chemfig{?(-[:190]H\O)-[:-50](-[:170]H\O)-[:10](-[:-55,0.7,,2]H|\O)-[:-10](-[6,0.7]\O H)-[:130]\O-[:190]?(-[:150,0.7]-[2,0.7]\O H)}

\chemfig[cram width=2pt]{\vpha|H|\O-[2,0.5,3]?<[7,0.7](-[2,0.5]\O H)-[,,,,
 line width=3pt](-[6,0.5]\O H)>[1,0.7](-[6,0.5]\O|H)-[3,0.7]
 \O-[4]?(-[2,0.3]-[3,0.5,,2]H|\O)}

 \chemfig[cram width=2pt]{H|\O-[-1,0.5]-[-2,0.4]?
 <[:-60,0.8](-[-2,0.5,,2]H|\O)
 -[,,,,line width=3pt](-[2,0.5,,2]H|\O)
 >[:60,0.8](-[6,0.5]\O|H)
 -[:150,1.03]\O?}












