\documentclass[leqno,a4paper,twoside,aps,prl,preprint,10pt,notitlepage]{revtex4-1}
\usepackage[head=12pt,includehead,inner=4cm,outer=1cm,top=1cm,bottom=1.5cm]{geometry}
\usepackage{parskip}
\usepackage[utf8]{inputenc}
\usepackage{chemfig}
\usepackage{longtable}
\usepackage{amssymb}
\usepackage{multirow}
\usepackage[verbose]{wrapfig}
\usepackage{textpos}
\usepackage{fancyhdr}
\usepackage{makecell}
\usepackage{array}
\usepackage{amsmath}
\usepackage{ifthen}
\usepackage{cleveref}
%\usepackage{titlesec}

\pagestyle{fancy}
\fancyheadoffset{0pt}
\renewcommand{\sectionmark}[1]{\markboth{#1}{}}
\fancyhf{}
\fancyhead[LE,RO]{\leftmark}
\fancyhead[RE,LO]{\thepage}
\fancypagestyle{plain}{%
  \fancyhead{} % get rid of headers
  \renewcommand{\headrulewidth}{0pt} % and the line
}

%\titlespacing\section{0pt}{15pt plus 4pt minus 2pt}{10pt plus 2pt minus 2pt}
%\titlespacing\subsection{0pt}{12pt plus 4pt minus 2pt}{8pt plus 2pt minus 2pt}
%\titlespacing\subsubsection{0pt}{8pt plus 4pt minus 2pt}{5pt plus 2pt minus 2pt}
\setlength{\TPHorizModule}{\textwidth}
\setlength{\TPVertModule}{\textheight}
%\setlength{\parskip}{0cm}

\let\stdsection\section
\renewcommand\section{\newpage\stdsection}

%\setlength{\parindent}{0em}
%\setlength{\parskip}{-1ex}
%\renewcommand{\baselinestretch}{1.0}

\usepackage{color}
\renewcommand\theequation{{{\bf{{\color{red}\arabic{equation}}}}}}
%\usepackage{mathtools}
%\newtagform{sq}{[}{]}

\makeatletter
\let\oldtagform@\tagform@
\renewcommand\tagform@[1]{\maketag@@@{\ignorespaces#1\unskip\@@italiccorr}}
\renewcommand{\eqref}[1]{\textup{\oldtagform@{\ref{#1}}}}
\makeatother


\usetikzlibrary{decorations.pathmorphing}



\newcommand{\vpha}{\vphantom{A}}
\newcommand{\degree}{\ensuremath{^\circ}}
\newcommand{\chemar}[2][]{
  \schemestart[{},#1]
  \vpha\arrow{#2}\vpha
  \schemestop
}
\newcommand{\chemeq}[3][]{
  \chemnameinit{}
  \noindent
  \begin{equation}
    \schemestart[][base]
    #3
    \schemestop
    %\hskip \textwidth minus \textwidth
    \ifthenelse{\equal{#1}{}}{}{\hskip 10mm\rm{also\ }\cref{#1}}
    \ifthenelse{\equal{#2}{}}{}{\label{#2}}
  \end{equation}
}
\newcommand{\precyc}{\vpha-[,0.1,,,draw=none]}

\newcommand{\grn}[1]{\vpha{\color{green}#1}}
\newcommand{\blu}[1]{\vpha{\color{blue}#1}}
\newcommand{\blk}[1]{\vpha{\color{black}#1}}


\makeatletter
\definearrow1{s>}{%
\ifx\@empty#1\@empty
\expandafter\draw\expandafter[\CF@arrow@current@style,-CF](\CF@arrow@start@node)--(\CF@arrow@end@node);%
\else
\def\curvedarrow@style{shorten <=\CF@arrow@offset,shorten >=\CF@arrow@offset,}%
\CF@expadd@tocs\curvedarrow@style\CF@arrow@current@style
\expandafter\draw\expandafter[\curvedarrow@style,-CF](\CF@arrow@start@name)..controls#1..(\CF@arrow@end@name);
\fi
}
\makeatother



%\let\origsection\section
%\makeatletter
%\renewcommand{\section}[1]{%
%
%    \vphantom{.}
%
%    \vspace{-5ex}
%    \origsection{#1}
%    \vspace{-4ex}
%}
%\makeatother
%
%\let\origsubsection\subsection
%\makeatletter
%\renewcommand{\subsection}[1]{%
%
%    \vphantom{.}
%
%    \vspace{-5ex}
%    \origsubsection{#1}
%    \vspace{-4ex}
%}
%\makeatother
%
\let\origsubsubsection\subsubsection
\makeatletter
\renewcommand{\subsubsection}[1]{%
    \vphantom{.}
    \vspace{-5ex}
    \origsubsubsection{\large\color{red}#1}
    \vspace{-4ex}
}
\makeatother




\begin{document}

\abovedisplayskip=9pt plus 3pt minus 9pt
\abovedisplayshortskip=0pt plus 3pt
\belowdisplayskip=9pt plus 3pt minus 9pt
\belowdisplayshortskip=4pt plus 3pt minus 4pt

\definesubmol\no{\vpha-[,0.1,,,draw=none]}
\definesubmol\Me[H_3C]{CH_3}
\definesubmol\Ammonia{\Lewis{4:,N}H_3}
\definesubmol\Cyanide{\quad\llap{${}^-$}\Lewis{4:,C}~\Lewis{0:,N}}
\definesubmol\Hydroxide{\quad\llap{${}^-$}OH}
\definesubmol\Water{H_2O}
\definesubmol\Alkene{C([:120]-)([:-120]-)=C([:-60]-)([:60]-)}
\newcommand{\X}{H}
\newcommand{\M}{\chemabove{\N}{\X}}
\newcommand{\K}{\chembelow{\N}{\X}}
\definesubmol\R{ \chemskipalign{\mbox{\scriptsize R}} }
\definesubmol{x}{(<[4]H)(<[0]O|H)}
\definesubmol{y}{(<[0]H)(<[4]H|O)}
\definesubmol{z}{CHOH}
\definesubmol{Z}{{(}CHOH{)}_n}

\setchemfig{atom sep=2em}
%\setcrambond{2pt}{0.75pt}{0.75pt}
\setchemfig{cram width = 2pt}
\setchemfig{cram dash width = 0.75pt}
\setchemfig{cram dash sep = 0.75pt}
\setchemfig{arrow label sep = 1.5pt}
\setchemfig{bond join=true}


\section{AkJKHfjhgFJHgfjhgf JHGF jhgf }

\subsection{sdfg afgdfg}

\subsubsection{kjasdf jhgasdf  kjhasdf kjh asdf}



\chemeq{ama2ald_1}{
  \chemnameinit{\chemfig{!\precyc*3(2-**6(--*5(-(=O)-(-[1]OH)(-[7]OH)-(=O)-)----))}}
  \chemname{\chemfig{NH_2-CH(-[2]R)-COOH}}
           {amino acid}
  \+
  \setchemfig{atom sep=1.4em}
  \chemname{\chemfig{!\precyc*3(2-[,,,,draw=none]**6(--*5(-(=O)-(-[1]OH)(-[7]OH)-(=O)-)----))}}
           {ninhydrin}
  \arrow(.mid east--.mid west) {->[pyridine]}[,1.2]
  \setchemfig{atom sep=1.3em}
  \chemname{\chemfig{!\precyc*3(-[,,,,draw=none]**6(--*5(-(=O)-(=N-*5(-(=O)-**6(------)--(-[,,,1]O|^{-})=))-(=O)-)----))}}
           {Ruhemann's purple}
  \arrow{0}[,0.25] \makecell[l]{\+ \chemfig{R-CHO}\\ \+ \chemfig{CO_2\uparrow}}
}


asdf \cref{ama2ald_1}


%\begin{preview}
%

\section{Synthesis of Alkyl Halides}


\subsection{From alkanes: free-radical halogenation \normalfont(Sections 4-13 and 6-6)}

\chemeq{kan2ah_1}{
  \chemfig{R-H}
  \chemar[1.5]{->[\chemfig{X_2}][heat or light]}
  \chemfig{R-X}
  \+
  \chemfig{H-X}
}


\subsection{From alkenes and alkynes \normalfont(Sections 6-6, 8-8, 9-9, 15-17)}

\chemeq[ken2ah_2,ken2ah_3]{ken2ah_1}{
  \chemfig{!\Alkene}
  \chemar{->[\chemfig{HX}]}
  \chemfig{C([2]-)([4]-)([6]-H)-C([0]-)([2]-)([6]-X)}
  \quad\text{(Sections 8-3, 8-8)}
}

\chemeq[ken2ah_5]{ken2ah_4}{
  \chemfig{!\Alkene}
  \chemar{->[\chemfig{X_2}]}
  \chemfig{C([2]-X)([4]-)([6]-)-C([0]-)([2]-)([6]-X)}
  \quad\text{(Sections 8-8)}
}

\chemeq[kyn2ken2ah_1]{kyn2ah_1}{
  \chemfig{C([4]-)~C([0]-)}
  \chemar{->[\chemfig{2X_2}]}
  \chemfig{C([2]-X)([4]-)([6]-X)-C([0]-)([2]-X)([6]-X)}
  \quad\text{(Section 9-9D)}
}

\chemeq[kyn2ken2ah_2]{kyn2ah_2}{
  \chemfig{C([4]-)~C([0]-)}
  \chemar{->[\chemfig{2HX}]}
  \chemfig{C([2]-H)([4]-)([6]-H)-C([0]-)([2]-X)([6]-X)}
  \quad\text{(Section 9-9E)}
}


\chemeq{ken2ah_6}{
  \chemfig{C([:120]-)([:-120]-)=C([:-60]-C([0]-H)([5]-)([6]-))([:60]-)}
  \chemar{->[\em{NBS}][light]}
  \chemfig{C([:120]-)([:-120]-)=C([:-60]-C([0]-Br)([5]-)([6]-))([:60]-)}
  \quad\text{(Sections 6-6, 15-7)}
}


\subsection{From alcohols \normalfont(Sections 11-7, 11-8, 11-9)}

\chemeq[al2ah_2]{al2ah_1}{
  \chemfig{R-OH}
  \chemar[1.2]{->[\chemfig{HX}, \chemfig{PX_3}][or others]}
  \chemfig{R-X}
}


\subsection{From other halides \normalfont(Section 6-9)}

\chemeq[ah2ah_2]{ah2ah_1}{
  \chemfig{R-X} + \chemfig{I^{-}}
  \chemar{->[\em{acetone}]}
  \chemfig{R-I} + \chemfig{X^{-}}
}

\chemeq{ah2ah_3}{
  \chemfig{R-Cl} + \chemfig{KF}
  \chemar[1.5]{->[\em{18-crown-6}][\chemfig{CH_3CN}]}
  \chemfig{R-F}
}



\section{Reactions of Alkyl Halides}

\subsection{Nucleophilic substitutions \normalfont(Section 6-9)}

\subsubsection{Alcohol formation}

\chemeq[ah2al_2]{ah2al_1}{
  \chemfig{R-X} + \chemfig{\quad\llap{${}^-$}\Lewis{2:4:6:,O}H}
  \chemar{->}
  \chemfig{R-OH} + \chemfig{\Lewis{0:2:4:6:,X}^{-}}
}

\subsubsection{Halide exchange}

\chemeq[ah2ah_1]{ah2ah_2}{
  \chemfig{R-X} + \chemfig{\Lewis{0:2:4:6:,I}^{-}}
  \chemar{->}
  \chemfig{R-I} + \chemfig{\Lewis{0:2:4:6:,X}^{-}}
}

\subsubsection{Williamson ether synthesis}

\chemeq{ah2ther_1}{
  \chemfig{R-X} + \chemfig{R|'\Lewis{0:2:6:,O}^{-}}
  \chemar{->}
  \chemfig{R-\Lewis{2:6:,O}-R|'} + \chemfig{\Lewis{0:2:4:6:,X}^{-}}
  \quad\text{ether synthesis}
}

\chemeq{ah2ther_2}{
  \chemfig{R-X} + \chemfig{R|'\Lewis{0:2:6:,S}^{-}}
  \chemar{->}
  \chemfig{R-\Lewis{2:6:,S}-R|'} + \chemfig{\Lewis{0:2:4:6:,X}^{-}}
  \quad\text{thioether synthesis}
}


\subsubsection{Amine synthesis}

\chemeq[ah2amn_2]{ah2amn_1}{
  \chemfig{R-X} + \chemname{\chemfig{\Ammonia}}{excess}
  \chemar{->}
  \chemfig{R-NH_3^{+}X^{-}}
  \chemar{->[\chemfig{\Ammonia}]}
  \chemfig{R-\Lewis{2:,N}H_2} + \chemfig{NH_4^{+}X^{-}}
}

\subsubsection{Nitrile synthesis}

\chemeq{ah2trile}{
  \chemfig{R-X} + \quad\chemfig{\llap{${}^-$}\Lewis{4:,C}~\Lewis{0:,N}}
  \chemar{->}
  \chemfig{R-C~N} + \chemfig{\Lewis{0:2:4:6:,X}^{-}}
}

\subsubsection{Alkyne synthesis \normalfont(Section 9-7A)}

\chemeq[ah2kyn_2,ah2kyn_3]{ah2kyn_1}{
  \chemfig{R-X} + \chemfig{R|'-C~\Lewis{0:,C}|\rlap{${}^-$}}
  \chemar{->}
  \chemfig{R-C~C-R|'} + \chemfig{\Lewis{0:2:4:6:,X}^{-}}
}


\subsection{Elimination}

\subsubsection{Dehydrohalogenation \normalfont(Sections 6-18 and 7-9A)}

\chemeq[ah2ken_2]{ah2ken_1}{
  \chemfig{C([2]-H)([4]-)([6]-)-C([0]-)([2]-)([6]-X)}
  \chemar{->[\chemfig{KOH}]}
  \chemfig{!\Alkene} + \chemfig{\Lewis{0:2:4:6:,X}^{-}}
}

\subsubsection{Dehalogenation \normalfont(Section 7-9D)}

\chemeq[ah2ken_4]{ah2ken_3}{
  \chemfig{C([2]-Br)([4]-)([6]-)-C([0]-)([2]-)([6]-Br)}
  \chemar{->[\chemfig{KI}]}
  \chemfig{!\Alkene} + \chemfig{I-Br} + \chemfig{K-Br}
}


\subsection{Formation of organometallic reagents \normalfont(Section 10-8)}

\subsubsection{Grignard reagents}

\chemeq{ah2gr}{
  \chemfig{R-X} + \chemfig{Mg}
  \chemar{->[ether]}
  \chemname{\hspace{10mm}\chemfig{R-\vpha Mg-X}\hspace{10mm}}
           {organomagnesium halide\\ (Grignard reagent)}
  \quad(X = Cl, Br or I)
}

\subsubsection{Organolithium reagents}

\chemeq{ah2om_1}{
  \chemfig{R-X} + \chemfig{2Li}
  \chemar{->}
  \chemname{\hspace{5mm}\chemfig{R-Li}\hspace{5mm}} {organolithium}
    + \chemfig{\hspace{5mm}Li^+X^{-}}
  \quad(X = Cl, Br or I)
}


\subsection{Coupling of organocopper reagents \normalfont(Section 10-9)}

\chemeq{om2om}{
  \chemfig{2R-Li} + \chemfig{CuI}
  \chemar{->}
  \chemfig{R_2CuLi} + \chemfig{LiI}
}

\chemeq{ah2kan_1}{
  \chemfig{R_2CuLi} + \chemfig{R|'-X}
  \chemar{->}
  \chemfig{R-R|'} + \chemfig{R-Cu} + \chemfig{LiX}
}


\subsection{Reduction \normalfont(Section 10-10)}

\chemeq{ah2kan_2}{
  \chemfig{R-X}
  \chemar[1.5]{->[(1) \chemfig{Mg} or \chemfig{Li}][(2) \chemfig{\Water}\hphantom{AAl}]}
  \chemfig{R-H}
}



\section{Synthesis of Alkenes}


\subsection{Dehydrohalogenation of alkyl halides \normalfont(Section 7-9)}

\chemeq[ah2ken_1]{ah2ken_2}{
  \chemfig{C([2]-H)([4]-)([6]-)-C([0]-)([2]-)([6]-X)}
  \chemar[1.3]{->[base][heat]}
  \chemfig{!\Alkene} + \chemfig{HX}
}


\subsection{Dehalogenation of vicinal dibromides \normalfont(Section 7-9D)}

\chemeq[ah2ken_3]{ah2ken_4}{
  \chemfig{C([2]-Br)([4]-)([6]-)-C([0]-)([2]-)([6]-Br)}
  \chemar{->[\chemfig{NaI}][aceton]}
  \chemfig{!\Alkene}
}


\subsection{Dehydration of alcohols \normalfont(Section 7-10)}

\chemeq[ken2al_2]{ken2al_1}{
  \chemfig{C([2]-)([4]-)([6]-H)-C([0]-)([2]-)([6]-OH)}
  \chemar[2.5]{->[conc. \chemfig{H_2SO_4} or \chemfig{H_3PO_4}][heat]}
  \chemfig{!\Alkene} + \chemfig{\Water}
}


\subsection{Dehydrogenation of alkanes \normalfont(Section 7-11B)}

\chemeq{kan2ken_1}{
  \chemfig{C([2]-)([4]-)([6]-H)-C([0]-)([2]-)([6]-H)}
  \chemar{->[catalyst][heat]}
  \chemfig{!\Alkene} + \chemfig{H_2}
}
(Industrial preparation, useful only for small alkenes; commonly gives mixtures.)


\subsection{Hofmann and Cope eliminations \normalfont(Sections 19-15 and 19-16)}

\chemeq[amn2ken_2]{amn2ken_1}{
  \chemfig{C([2]-H)([4]-)([6]-)-C([0]-)([2]-)([6]-[,,,2]
    {\llap{${}^+$}}N|{(CH_3)_3\hspace{3mm}I^{-}})}
  \chemar{->[\chemfig{Ag_2O}][heat]}
  \chemfig{!\Alkene} + \chemfig{\Lewis{4:,N}|{(CH_3)_3}}
}
(Usually gives the least substituted alkene.)


\subsection{Reduction of alkynes \normalfont(Section 9-9)}

\chemeq[kyn2ken_2]{kyn2ken_1}{
  \chemfig{R-C~C-R|'}
  \chemar[1.7]{->[\chemfig{H_2}, \chemfig{Pd}/\chemfig{NaSO_4}][quinoline]}
  \chemfig{C([:120]-R)([:-120]-H)=C([:-60]-R|')([:60]-H)}
  cis \quad\text{(Section 9-9A)}
}

\chemeq[kyn2ken_4]{kyn2ken_3}{
  \chemfig{R-C~C-R|'}
  \chemar[1.2]{->[\chemfig{Na}, \chemfig{NH_3}]}
  \chemfig{C([:120]-R)([:-120]-H)=C([:-60]-R|')([:60]-H)}
  trans \quad\text{(Section 9-9B)}
}


\subsection{Wittig reaction \normalfont(Section 18-13)}

\chemeq{ket2ken_1}{
  \chemfig{C([:120]-[,,,1]R|')([:-120]-R)=O} + \chemfig{Ph_3P=CHR|''}
  \chemar{->}
  \chemfig{C([:120]-[,,,1]R|')([:-120]-R)=R|''} + \chemfig{Ph_3P=CHO}
}



\section{Reactions of Alkenes}


\subsection{Electrophilic Additions}

\subsubsection{Addition of hydrogen halides \normalfont(Section 8-3)}

\chemeq[ken2ah_1]{ken2ah_2}{
  \chemfig{!\Alkene}
  \+
  \chemfig{H-X}
  \chemar{->}
  \chemfig{C([2]-)([4]-)([6]-H)-C([0]-)([2]-)([6]-X)}
  \quad Markovnikov\ orientation
}

(X = Cl, Br or I)

\chemeq[ken2ah_1]{ken2ah_3}{
  \chemfig{!\Alkene} + \chemfig{H-X}
  \chemar[1.3]{->[peroxides]}
  \chemfig{C([2]-)([4]-)([6]-H)-C([0]-)([2]-)([6]-X)}
  \quad anti-Markovnikov\ orientation
}

\subsubsection{Acid-catalyzed hydration \normalfont(Section 8-4)}

\chemeq[ken2col_2]{ken2col_1}{
  \chemfig{!\Alkene} + \chemfig{\Water}
  \chemar{->[\chemfig{H^+}]}
  \chemfig{C([2]-)([4]-)([6]-H)-C([0]-)([2]-)([6]-OH)}
  \quad Markovnikov\ orientation
}

\subsubsection{Oxymercuration-demercuration \normalfont(Section 8-5)}

\chemeq[ken2col_4]{ken2col_3}{
  \chemfig{!\Alkene} + \chemfig{Hg{(}OAc{)}_2}
  \chemar{->[\chemfig{\Water}]}
  \chemfig{C([2]-)([4]-)([6]-[,,,2]HO)-C([0]-)([2]-)([6]-HgOAc)}
  \chemar{->[\chemfig{NaBH_4}]}
  \chemname{\hspace{5mm}\chemfig{C([2]-)([4]-)([6]-[,,,2]HO)-C([0]-)([2]-)([6]-H)}\hspace{5mm}}
           {Markovnikov orientation}
}

\subsubsection{Alkoxymercuration-demercuration \normalfont(Section 8-6)}

\chemeq[kenal2est_2]{kenal2est_1}{
  \chemfig{!\Alkene} + \chemfig{Hg{(}OAc{)}_2}
  \chemar{->[\chemfig{R-OH}]}
  \chemfig{C([2]-)([4]-)([6]-[,,,2]RO)-C([0]-)([2]-)([6]-HgOAc)}
  \chemar{->[\chemfig{NaBH_4}]}
  \chemname{\hspace{5mm}\chemfig{C([2]-)([4]-)([6]-[,,,2]RO)-C([0]-)([2]-)([6]-H)}\hspace{5mm}}
           {Markovnikov orientation}
}

\subsubsection{Hydroboration-oxidation \normalfont(Section 8-7)}

\chemeq[ken2col_6]{ken2col_5}{
  \chemfig{!\Alkene} + {\em BH$_3\cdot$THF}
  \chemar{->}
  \chemfig{C([2]-)([4]-)([6]-H)-C([0]-)([2]-)([6]-BH_2)}
  \chemar{->[\chemfig{H_2O_2}][\chemfig{\llap{${}^-$}OH}]}
  \chemname{\hspace{5mm}\chemfig{C([2]-)([4]-)([6]-H)-C([0]-)([2]-)([6]-OH)}\hspace{5mm}}
           {anti-Markovnikov orientation\\(syn addition)}
}

\subsubsection{Polymerization \normalfont(Section 8-16)}

\chemeq{ken2pol_1}{
  \chemfig{R|^+} + {\setchemfig{chemfig style={blue}}\chemfig{!\Alkene}}
  \chemar{->}
  {\setchemfig{chemfig style={blue}}
    \chemfig{\blk{R}-C([2]-)([6]-)-C{\rlap{${}^+$}}([:60]-)([:-60]-)}}
  \chemar{->[{\setchemfig{atom sep=1.5em}\chemfig{!\Alkene}}]}
  \chemfig{R-[,,,,blue]\blu{C}([2]-[,,,,blue])([6]-[,,,,blue])
           -[,,,,blue]\blu{C}([2]-[,,,,blue])([6]-[,,,,blue])
           -C([2]-)([6]-)-C{\rlap{${}^+$}}([:60]-)([:-60]-)}
}
(also radical and anionic polymerization)


\subsection{Reduction: Catalytic Hydrogenation \normalfont(Section 8-10)}

\chemeq{ken2kan_1}{
  \chemfig{!\Alkene} + \chemfig{H_2}
  \chemar[1.5]{->[Pt, Pd or Ni]}
  \chemfig{C([2]-)([4]-)([6]-H)-C([0]-)([2]-)([6]-H)}
  \quad(syn addition)
}


\subsection{Addition of Carbenes: Cyclopropanation \normalfont(Section 8-11)}

\chemeq{ken2cyc_1}{
  \chemfig{!\Alkene} + \chemfig{\Lewis{4:,C}([:60]-X)([:-60]-Y)}
  \chemar{->}
  \chemfig{C([:-60]-X)([:-120]-Y)([:60]-C?([0]-)([2]-))([:120]-C?([2]-)([4]-))}
  \quad(X,Y = H, Cl, Br, I, or \chemfig{\vpha-COOEt})
}


\subsection{Oxidative Additions}

\subsubsection{Addition of halogens \normalfont(Section 8-8)}

\chemeq[ken2ah_4]{ken2ah_5}{
  \chemfig{!\Alkene} + \chemfig{X_2}
  \chemar{->}
  \chemname{\chemfig{C([2]-X)([4]-)([6]-)-C([0]-)([2]-)([6]-X)}}
           {(anti addition)}
  \quad (X = Cl, Br, sometimes I)
}

\subsubsection{Halohydrin formation \normalfont(Section 8-9)}

\chemeq{ken2hal_1}{
  \chemfig{*6(--=(-)---)} \arrow{0}[,0]\+ \chemfig{Br_2}
  \arrow{->[\chemfig{\Water}]}
  \chemfig{*6(--([0]<Br)([7]>:H)-([1]<)([0]>:OH)---)}
}
\hspace{2mm} anti addition (Markovnikov orientation)

\subsubsection{Epoxidation \normalfont(Sections 8-12 and 14-11A)}

\chemeq[ken2epx_2]{ken2epx_1}{
  \chemfig{!\Alkene} + \chemfig{R-C([6]=O)-O-O-H}
  \chemar{->}
  \chemfig{C([2]-)([4]-)([:-60]-O?)-C?([0]-)([2]-)} + \chemfig{R-C([6]=O)-O-H}
}

\subsubsection{Anti hydroxylation \normalfont(Section 8-13)}

\chemeq[ken2col_11]{ken2col_10}{
  \chemfig{!\Alkene} + \chemfig{R-C([6]=O)-O-O-H}
  \chemar[0.8]{->}
  \chemfig{C([2]-)([4]-)([:-60]-O?)-C?([0]-)([2]-)}
  \chemar[0.8]{->[\chemfig{H^+}][\chemfig{\Water}]}
  \chemfig{C([2]-OH)([4]-)([6]-)-C([0]-)([2]-)([6]-OH)}
}

\subsubsection{Syn hydroxylation \normalfont(Section 8-14)}


\chemeq[ken2col_9]{ken2col_7}{
  \chemfig{!\Alkene} + \chemfig{KMnO_4}
  \chemar[1.7]{->[\chemfig{!\Hydroxide}, \chemfig{!\Water}][cold, delute]}
  \chemfig{C([2]-)([4]-)([6]-[,,,2]HO)-C([0]-)([2]-)([6]-OH)}
}

\chemeq[ken2col_9]{ken2col_8}{
  \chemfig{!\Alkene} + \chemfig{OsO_4}
  \chemar{->[\chemfig{H_2O_2}]}
  \chemfig{C([2]-)([4]-)([6]-[,,,2]HO)-C([0]-)([2]-)([6]-OH)}
}


\subsection{Oxidative Cleavage of Alkenes \normalfont(Section 8-15)}

\subsubsection{Ozonolysis}

\chemeq[ken2ahket_2]{ken2ahket_1}{
  \chemname{\chemfig{C([:120]-R)([:-120]-H)=C([:-60]-R|')([:60]-R|'')}}
           {alkene}
  \chemar{->[\chemfig{O_3}]}
  \chemname{\chemfig{[:-18]C([3]-R)([5]-H)*5(-O-O-C([1]-R|'')([7]-R|')-O-)}}
           {ozonide}
  \chemar[1.2]{->[\chemfig{{(}CH_3{)}_2S}]}
  \chemname{\chemfig{C([:120]-R)([:-120]-H)=O}}
         {aldehyde} +
  \chemname{\chemfig{O=C([:-60]-R|')([:60]-R|'')}}
         {ketone}
}

\subsubsection{Potassium permanganate}

\chemeq[ken2ket2acid_2]{ken2ket2acid_1}{
  \chemfig{C([:120]-R)([:-120]-R)=C([:-60]-H)([:60]-R|')} + \chemfig{KMnO_4}
  \chemar{->[warm]}
  \chemname{\chemfig{C([:120]-R)([:-120]-R)=O}}{ketones} +
  \chemname{\chemfig{O=C([:-60]-OH)([:60]-R|')}}{acids}
  \quad\parbox{3.5cm}{(aldehydes are oxidized to acids)}
}


\subsection{Olefin (Alkene) Metathesis \normalfont(Section 8-17)}

\chemeq{ken2ken_1}{
  \chemfig{C([:120]-[,,,1]R|^1)([:-120]-H)=C([:-60]-H)([:60]-H)} +
  \chemfig{C([:120]-[,,,1]R|^2)([:-120]-H)=C([:-60]-H)([:60]-H)}
  \chemar[1.5]{<<->[catalyst]}
  \chemname{\chemfig{C([:120]-[,,,1]R|^1)([:-120]-H)=C([:-60]-R|^2)([:60]-H)}}{(cis + trans)} +
  \chemname{\chemfig{C([:120]-H)([:-120]-H)=C([:-60]-H)([:60]-H)}}{ethylene}
}



\section{Syntheses of Alkynes}


\subsection{Alkylation of acetylide ions \normalfont(Section 9-7A)}

\chemeq[ah2kyn_1,ah2kyn_3]{ah2kyn_2}{
  \chemfig{R-C~\Lewis{0:,C}|^{-}} + \chemfig{R|'-X}
  \chemar{->[S\textsubscript{N}2]}
  \chemfig{R-C~C-R|'} + \chemfig{X^{-}}
}
(\chemfig{R|'-X} must be an unhindered primary halide or tosylate)


\subsection{Additions to carbonyl groups \normalfont(Section 9-7B)}

\chemeq[cnyl2kyn_2,cnyl2kyn_3]{cnyl2kyn_1}{
  \chemfig{R-C~C|^{-}} + \chemfig{C([:120]-[,,,1]R|')([:-120]-[,,,1]R|')=O}
  \chemar{->}
  \chemar[1.1]{->[\chemfig{\Water}][or \chemfig{H_3O^+}]}
  \chemfig{R-C~C-C([2]-R|')([6]-R|')-OH}
}


\subsection{Double dehydrohalogenation of alkyl dihalides \normalfont(Section 9-8)}

\chemeq{ah2kyn_4}{
  \chemfig{R-C([2]-X)([6]-H)-C([2]-X)([6]-H)-R|'}
  \quad or\quad
  \chemfig{R-C([2]-H)([6]-H)-C([2]-X)([6]-X)-R|'}
  \chemar[2.5]{->[fused \chemfig{KOH} or \chemfig{NaNH_2}][(severe conditions)]}
  \chemfig{R-C~C-R|'}
}
(\chemfig{KOH} forms internal alkynes; \chemfig{NaNH_2} forms terminal alkynes.)



\section{Reactions of Alkynes}

\noindent
ACETYLIDE CHEMISTRY


\subsection{Formation of acetylide anions (alkynides) \normalfont(Section 9-6)}

\chemeq{kyn2knd_1}{
  \chemfig{R-C~C-H} + \chemfig{NaNH_2}
  \chemar{->}
  \chemfig{R-C~\Lewis{0:,C}|^{-}|^+Na} + \chemfig{NH_3}
}

\chemeq{kyn2knd_2}{
  \chemfig{R-C~C-H} + \chemfig{R|'MgX}
  \chemar{->}
  \chemfig{R-C~C-MgX} + \chemfig{R|'H}
}

\chemeq{kyn2knd_3}{
  \chemfig{R-C~C-H} + \chemfig{R|'Li}
  \chemar{->}
  \chemfig{R-C~C-Li} + \chemfig{R|'H}
}


\subsection{Alkylation of acetylide ions \normalfont(Section 9-7A)}

\chemeq[ah2kyn_1,ah2kyn_2]{ah2kyn_3}{
  \chemfig{R-C~\Lewis{0:,C}|^{-}} + \chemfig{R|'-X}
  \chemar{->}
  \chemfig{R-C~C-R|'}
}
(\chemfig{R|'-X} must be an unhindered primary halide or tosylate.)


\subsection{Reactions with carbonyl groups \normalfont(Section 9-7B)}

\chemeq[cnyl2kyn_1,cnyl2kyn_3]{cnyl2kyn_2}{
  \chemfig{R-C~C|^{-}} + \chemfig{C([:120]-[,,,1]R|')([:-120]-[,,,1]R|')=O}
  \chemar[1.1]{->}
  \chemar{->[\chemfig{\Water}][or \chemfig{H_3O^+}]}
  \chemfig{R-C~C-C([2]-R|')([6]-R|')-OH}
}

\vspace{10mm}\noindent
ADDITIONS TO THE TRIPLE BOND

\subsection{Reduction to alkanes \normalfont(Sections 9-9A)}
\chemeq{kyn2kan_1}{
  \chemfig{R-C~C-R|'} + 2\chemfig{H_2}
  \chemar[1.3]{->[Pt, Pd or Ni]}
  \chemfig{R-C([2]-H)([6]-H)-C([2]-H)([6]-H)-R|'}
}


\subsection{Reduction to alkanes \normalfont(Sections 9-9B and 9-9C)}

\chemeq[kyn2ken_1]{kyn2ken_2}{
  \chemfig{R-C~C-R|'} + \chemfig{H_2}
  \chemar[2.3]{->[\chemfig{Pd}/\chemfig{BaSO_4}, quinoline][(Lindlar's catalyst)]}
  \chemfig{C([:120]-R)([:-120]-H)=C([:-60]-H)([:60]-R|')}
  cis
}

\chemeq[kyn2ken_3]{kyn2ken_4}{
  \chemfig{R-C~C-R|'}
  \chemar[1.2]{->[\chemfig{Na}, \chemfig{NH_3}]}
  \chemfig{C([:120]-R)([:-120]-H)=C([:-60]-R|')([:60]-H)}
  trans
}


\subsection{Addition of halogens \normalfont(Section 9-9D)}

\chemeq[kyn2ah_1]{kyn2ken2ah_1}{
  \chemfig{R-C~C-R|'}
  \chemar{->[\chemfig{X_2}]}
  \chemname{\chemfig{C([:120]-R)([:-120]-X)=C([:-60]-R|')([:60]-X)}}{cis + trans}
  \chemar{->[\chemfig{X_2}]}
  \chemfig{R-C([2]-X)([6]-X)-C([2]-X)([6]-X)-R|'}
  (X = Cl, Br)
}


\subsection{Addition of hydrogen halides \normalfont(Section 9-9E)}

\chemeq[kyn2ah_2]{kyn2ken2ah_2}{
  \chemfig{R-C~C-R|'}
  \chemar{->[\chemfig{HX}]}
  \chemname{\chemfig{C([:120]-R)([:-120]-H)=C([:-60]-R|')([:60]-X)}}
           {Markovnikov orientation}
  \chemar{->[\chemfig{HX}]}
  \chemfig{R-C([2]-H)([6]-H)-C([2]-X)([6]-X)-R|'}
  (X = Cl, Br, I)
}


\subsection{Addition of water \normalfont(Section 9-9F)}

\subsubsection{Catalyzed by $HgSO_4$/$H_2SO_4$}

\chemeq[kyn2ket_2]{kyn2ket_1}{
  \chemfig{R-C~C-H} + \chemfig{\Water}
  \chemar{->[\chemfig{HgSO_4},][\chemfig{H_2SO_4}]}
  \chemname{\chemleft[
            \chemfig{C([:120]-R)([:-120]-[,,,2]HO)=C([:-60]-H)([:60]-H)}
            \chemright]}
           {vinyl alcohol\\(unstable)}
  \chemar{<->>}
  \chemname{\chemfig{R-C([:60]-CH_3)([:-60]=O)}}{ketone\\(stable)}
}
\vspace{-10mm}

(Markovnikov orientation)

\subsubsection{Hydroboration-oxidation}

\chemeq[kyn2ald_2]{kyn2ald_1}{
  \chemfig{R-C~C-H}
  \chemar[2]{->[(1) \chemfig{Sia_2BH\cdot THF}][(2) \chemfig{H_2O_2}, \chemfig{NaOH}]}
  \chemname{\chemleft[
            \chemfig{C([:120]-R)([:-120]-H)=C([:-60]-H)([:60]-OH)}
            \chemright]}
           {vinyl alcohol\\(unstable)}
  \chemar{<->>}
  \chemname{\chemfig{R-C([2]-H)([6]-H)-C([:60]=O)([:-60]-H)}}
           {aldehyde\\(stable)}
}
\vspace{-10mm}

(anti-Markovnikov orientation)


\vspace{10mm}\noindent
OXIDATION OF ALKYNES (Section 9-10)


\subsection{Oxidation to $\alpha$-diketones \normalfont(Section 9-10A)}

\chemeq{kyn2ket_3}{
  \chemfig{R-C~C-R|'}
  \chemar{->[\chemfig{KMnO_4}][\chemfig{\Water}]}
  \chemfig{C([:120]=O)([:-120]-R)-C([:-60]-R|')([:60]=O)}
}


\subsection{Oxidative cleavage \normalfont(Section 9-10B)}

\chemeq[kyn2acid_2]{kyn2acid_1}{
  \chemfig{R-C~C-R|'}
  \chemar[3]{->[(1) \chemfig{KMnO_4}, \chemfig{\Hydroxide} (2) \chemfig{H^+}][(or (1) \chemfig{O_3}, (2) \chemfig{\Water})]}
  \chemfig{R-C([:60]-OH)([:-60]=O)} +
  \chemfig{C([:120]-OH)([:-120]=O)-R|'}
}



\section{Synthesis of Alcohol}

\noindent
NUCLEOPHILIC SUBSTITUTION ON AN ALKYL HALIDE (Chapter 6)

Usually via the S\textsubscript{N}2 mechanism; competes with elimination.

\chemeq[ah2al_1]{ah2al_2}{
  {\setchemfig{chemfig style={blue}}
    \chemfig{H@{f1}\Lewis{0:2:6:,O}^{-}}}
  +
  \chemfig{@{f2}C([3]-R)([5]>:H)([:-100]<H)-[@{g1},,,,green]@{g2}\grn{X}}
  \chemmove{\draw[->,red](f1)..controls +(60:10mm) and +(180:10mm)..(f2);}
  \chemmove{\draw[->,red](g1)..controls +(-90:5mm) and +(-90:5mm)..(g2);}
  \chemar[0.8]{->}
  \chemname{\chemleft[
  \chemfig{\blu{H}|\blu{O}
           -[,,,,blue,dash pattern=on 1pt off 1pt]
           C([2]-R)([:-110]>:H)([:-70]<H)
           -[,,,,green,dash pattern=on 1pt off 1pt]
           {\color{green}X}}
  \chemright]\raisebox{6mm}{$\stackrel{-}{\ddagger}$}  } {transition state}
  \chemar[0.8]{->}
  \chemfig{\blu{H}|\blu{O}
           -[,,,,blue,dash pattern=on 1pt off 1pt]
           C([2]-R)([:-45]>:H)([:-80]<H) }
  + {\setchemfig{chemfig style={green}} \chemfig{^{-}X}}
}



\vspace{10mm}\noindent
SYNTHESIS OF ALCOHOLS FROM ALKENES (Chapter 8)


\subsection{Acid-catalyzed hydration \normalfont(Section 8-4)}

\chemeq[ken2col_1]{ken2col_2}{
  \chemfig{!\Alkene} + {\setchemfig{chemfig style={green}}\chemfig{\Water}}
  \chemar{->[\chemfig{H^{+}}]}
  \chemfig{C([2]-)([4]-)([6]-\textcolor{green}{H})-C([0]-)([2]-)([6]-{\color{green}O}|{\color{green}H})}
  Markovnikov orientation
}


\subsection{Oxymercuration-demercuration \normalfont(Section 8-5)}

\chemeq[ken2col_3]{ken2col_4}{
  \chemfig{!\Alkene} + {\setchemfig{chemfig style={green}}\chemfig{Hg{(}OAc{)}_2}}
  \chemar{->[{\setchemfig{chemfig style={blue}}\chemfig{\Water}}]}
  \chemfig{C([2]-\blu{O}|\blu{H})([4]-)([6]-)
           -C([0]-)([2]-)([6]-{\color{green}Hg}|{\color{green}OAc})}
  \chemar{->[\chemfig{NaBH_4}]}
  \chemname{\chemfig{C([2]-\blu{O}|\blu{H})([4]-)
                     ([6]-)-C([0]-)([2]-)([6]-\blu{H})}}
           {Markovnikov orientation}
}


\subsection{Hydroboration-oxidation \normalfont(Section 8-7)}

\chemeq[ken2col_5]{ken2col_6}{
  \chemfig{!\Alkene}
  \chemar[2]{->[(1){\em BH$_3\cdot$THF}][(2)\chemfig{H_2O_2}, \chemfig{NaOH}]}
  \chemname{\hspace{5mm}\chemfig{C([2]-)([4]-)([6]-H)-C([0]-)([2]-)([6]-OH)}\hspace{5mm}}
           {syn addition, anti-Markovnikov orientation}
}


\subsection{Hydroxylation: synthesis of 1,2-diols from alkenes}

\subsubsection{Syn hydroxylation \normalfont(Section 8-14)}

\chemeq[ken2col_7,ken2col_8]{ken2col_9}{
  \chemfig{!\Alkene}
  \chemar[3]{->[\chemfig{KMnO_4}, \chemfig{!\Hydroxide}, (cold, delute)][or \chemfig{OsO_4}, \chemfig{H_2O_2}]}
  \chemfig{C([2]-)([4]-)([6]-[,,,2]HO)-C([0]-)([2]-)([6]-OH)}
}

\subsubsection{Anti hydroxylation \normalfont(Section 8-13)}

\chemeq[ken2col_10]{ken2col_11}{
  \chemfig{!\Alkene} +
  \chemar[2.2]{->[\chemfig{R-C([2]=O)-O-O-H}][\chemfig{H^+}, \chemfig{\Water}]}
  \chemfig{C([2]-OH)([4]-)([6]-)-C([0]-)([2]-)([6]-OH)}
}


\subsection{Additions of acetylides to carbonyl compaunds \normalfont(Section 9-7)}

\chemeq[cnyl2kyn_1,cnyl2kyn_2]{cnyl2kyn_3}{
  \chemnameinit{\chemfig{R-C~C-C(-[2]R|^{'})(-[6]R|^{'})-OH}}
  \chemname{\chemfig{R-C~\charge{0=\:,45=$\scriptstyle-$}{C}\phantom{i}@{x1}\vpha}}{acetylide}
  \+
  \chemname{\chemfig[chemfig style=green]{@{x2}C(-[:-120,,,1]R|^{'})(-[:120,,,1]R|^{'})=[@{y1}]\charge{45=\:,-45=\:}{O}@{y2}}}{ketone or aldehyde}
  \chemmove{\draw[->,red](x1)..controls +(60:8mm) and +(160:6mm)..(x2);}
  \chemmove{\draw[->,red](y1)..controls +(-90:4mm) and +(-90:4mm)..(y2);}
  \arrow{->}
  \chemname{
    \chemfig{R-C~C-\grn{C}
                   (-[2,,,,green]\grn{R}|^{\grn{'}})
                   (-[6,,,,green]\grn{R}|^{\grn{'}})
            -[,,,,green]\charge{-90={\color{green}\:},0={\color{green}\:},90={\color{green}\:},45=$\color{green}\scriptstyle-$}{\grn{O}}}
  }
  {alkoxide}
  \arrow{->[\chemfig{H_3O^+}]}
  \chemname{\chemfig{R-C~C-\grn{C}
                      (-[2,,,,green]\grn{R}|^{\grn{'}})
                      (-[6,,,,green]\grn{R}|^{\grn{'}})
                      -[,,,,green]\grn{O}H}}
            {accetylenic alcohol}
}

\vspace{3mm}\noindent
ALCOHOL SYNTHESES BY NUCLEOPHILIC ADDITIONS TO CARBONYL GROUPS

\subsection{Addition of a Grignard or organolithium reagent \normalfont(Section 10-9)}
Here and below in this section \chemfig{\vpha MgX} can be replaced with \chemfig{Li}.

\chemeq[ket2al_2]{ket2al_1}{
  \chemfig{R-MgX} + \chemfig{C([:120]-)([:-120]-)=O}
  \chemar{->[ether]}
  \chemfig{R-C([2]-)([6]-)-O|^{-}|^{+}MgX}
  \chemar{->[\chemfig{H_3O^{+}}]}
  \chemfig{R-C([2]-)([6]-)-OH}
}

\chemeq{om2al_1}{
  2\hspace{1mm}\chemfig{R-MgX} +
  \chemleft(
    \chemname{\chemfig{C([:120]-R\rlap{$'$})([:-120]-O-[:180])=O}}
             {ester}
    \qquad or \qquad
    \chemname{\chemfig{C([:120]-R\rlap{$'$})([:-120]-X)=O}\hspace{5mm}}
             {acid halide}
  \chemright)
  \chemar[1.2]{->[(1) ether][(2) \chemfig{H_3O^{+}}]}
  \chemfig{R|'-C([2]-R)([6]-R)-OH}
}

\chemeq{epx2al_1}{
  \chemfig{R-MgX} +
  \chemname{\hspace{5mm}\chemfig{[:-60]O*3(---)}\hspace{5mm}}
           {ethylene oxide}
  \chemar[1.2]{->[(1) ether][(2) \chemfig{H_3O^{+}}]}
  \chemfig{R-[:30]-[:-30]-[:30]OH}
}


\subsection{Reduction of carbonyl compounds \normalfont(Section 10-11)}

\subsubsection{Catalytic hydrogenation of aldehydes and ketones}

\chemeq[ket2al_5]{ket2al_3}{
  \chemfig{C([:120]-)([:-120]-)=O} + \chemfig{H_2}
  \chemar[1.2]{->[Raney \chemfig{Ni}]}
  \chemfig{C([:120]-)([:-120]-)-OH}
}
This method is usually not as selective or as effective as the use of hydride reagents.

\subsubsection{Use of hydride reagents}

\chemeq[ket2al_5]{ket2al_4}{
  \chemfig{C([:120]-)([:-120]-)=O}
  \chemar{->[\chemfig{NaBH_4}]}
  \chemfig{HC([2,,2]-)([6,,2]-)-OH}
}

\chemeq[acid0est2alco_2]{acid0est2alco_1}{
  \chemname{\chemfig{C([:120]-R)([:-120]-O-[:180])=O}}
           {acid or ester}
  \chemar[1.2]{->[(1) \chemfig{LiAlH_4}][(2) \chemfig{H_3O^{+}}]}
  \chemfig{R-[:30]-[:-30]OH}
}

\subsection{Acid-catalyzed oxide opening \normalfont(Sections 8-13 and 14-12)}
\chemeq[pox2al_2]{pox2al_1}{
  \chemfig{\vpha-C([6]-)([:60]-O?)-C?([6]-)-}
  \chemar{->[\chemfig{H^+}][\chemfig{R-OH}]}
  \chemfig{\vpha-C([2]-)([6]-OH)-C([2]-OR)([6]-)-}
  \qquad R can be hydrogen
}
The alkoxy group bonds to the more highly substituted carbon.



%\section{S\textsubscript{N}2 Reactions of Tosylate Esters \normalfont(Section 11-6)}
%
%\noindent
%\begin{tabular}{@{\hspace{-4pt}}lccclcc}
% %% TODO on map %%
%\chemnameinit{}
%\chemfig{R-OTs} & + & \chemname{\chemfig{\Hydroxide}}{hydroxide} &
%\chemar{->} &
%\chemname{\chemfig{R-OH}}{alcohol} & + & \chemfig{\quad\llap{${}^-$}OTs} \\
%\\
%\chemfig{R-OTs} & + & \chemname{\chemfig{^{-}C~N}}{cyanide} &
%\chemar{->} &
%\chemname{\chemfig{R-C~H}}{nitrile} & + & \chemfig{\quad\llap{${}^-$}OTs} \\
%\\
%\chemfig{R-OTs} & + & \chemname{\chemfig{Br^{-}}}{halide} &
%\chemar{->} &
%\chemname{\chemfig{R-Br}}{alkyl halide} & + & \chemfig{\quad\llap{${}^-$}OTs} \\
%\\
%\chemfig{R-OTs} & + & \chemname{\chemfig{R|'-O^{-}}}{alkoxide} &
%\chemar{->} &
%\chemname{\chemfig{R-O-R|'}}{ether} & + & \chemfig{\quad\llap{${}^-$}OTs} \\
%\\
%\chemfig{R-OTs} & + & \chemname{\chemfig{\Ammonia}}{ammonia} &
%\chemar{->} &
%\chemname{\chemfig{R-NH_3^{+}|^{-}OTs}}{amine salt} & & \\
%\\
%\chemfig{R-OTs} & + & \chemname{\chemfig{LiAlH_4}}{LAH} &
%\chemar{->} &
%\chemname{\chemfig{R-H}}{alkane} & + & \chemfig{\quad\llap{${}^-$}OTs}
%
%\end{tabular}
%



\section{Reactions of Alcohols}


\subsection{Oxidation-reduction reactions}

\subsubsection{Oxidation of secondary alcohols to ketones \normalfont(Section 11-2A)}

\chemeq[al2ket_2]{al2ket_1}{
  \chemfig{H-C([2]-R\rlap{$'$})([6]-R\rlap{$''$})-OH}
  \chemar[1.5]{->[\chemfig{Na_2Cr_2O_7}][\chemfig{H_2SO_4}]}
  \chemfig{C([:120]-R\rlap{$'$})([:-120]-R\rlap{$''$})=O}
}

\subsubsection{Oxidation of primary alcohols to carboxylic acids \normalfont(Section 11-2B)}

\chemeq{al2acid_1}{
  \chemfig{R-[:30]-[:-30]OH}
  \chemar[1.3]{->[\chemfig{Na_2Cr_2O_7}][\chemfig{H_2SO_4}]}
  \chemfig{C([:120]-R)([:-120]-HO)=O}
}

\subsubsection{Oxidation of primary alcohols to aldehydes \normalfont(Section 11-2B)}

\chemeq[al2ald_2]{al2ald_1}{
  \chemfig{R-[:30]-[:-30]OH}
  \chemar{->[\em{PCC}]}
  \chemfig{C([:120]-R)([:-120]-H)=O}
}

\subsubsection{Reduction of alcohols to alkanes \normalfont(Section 11-6)}

\chemeq{al2kan_1}{
  \chemfig{R-OH}
  \chemar[2]{->[(1)\chemfig{TsCl}/pyridine][(2)\chemfig{LiAlH_4}]}
  \chemfig{R-H}
}


\subsection{Cleavage of alcohol hydroxyl group.}

\subsubsection{Conversion of alcohols to alkyl halides \normalfont(Sections 11-7 - 11-9)}

\chemeq[al2ah_1]{al2ah_2}{
  \chemfig{R-OH}
  \chemar[1.2]{->[\chemfig{HX}, \chemfig{PX_3}][or others]}
  \chemfig{R-X}
}

\subsubsection{Dehydration of alcohols to form alkenes \normalfont(Section 11-10A)}

\chemeq[ken2al_1]{ken2al_2}{
  \chemfig{-C([2]-H)([6]-)-C([2]-OH)([6]-)-}
  \chemar[1.4]{<->[\chemfig{H_2SO_4}][or \chemfig{H_2PO_4}]}
  \chemfig{!\Alkene} + \chemfig{\Water}
}

\subsubsection{Industrial dehydration of alcohols to form ethers \normalfont(Section 11-10B)}

\chemeq[al2eth_2]{al2eth_1}{
  2\chemfig{R-OH}
  \chemar{<->[\chemfig{H^{+}}]}
  \chemfig{R-O-R} + \chemfig{\Water}
}


\subsection{Cleavage of the hydroxyl proton}

\subsubsection{Tosylation \normalfont(Section 11-5)}

\chemeq{al2aTs_1}{
  \chemfig{R-OH} + \chemfig{Cl-S([2]=O)([6]=O)-**6(---(-CH_3)---)}
  \chemar{->[pyridine]}
  \chemfig{R-O-S([2]=O)([6]=O)-**6(---(-CH_3)---)} + \chemfig{HCl}
}

\subsubsection{Acylation to form esters \normalfont(Section 11-12)}

\chemeq{al2est_3}{
  \chemfig{R-OH} + \chemfig{R|'-C([2]=O)-Cl}
  \chemar{->}
  \chemfig{R-C([2]=O)-O-R|'} + \chemfig{HCl}
}

\subsubsection{Deprotonation to form an alkoxide \normalfont(Section 11-14)}

\chemeq{al2axd_1}{
  \chemfig{R-OH} + \chemfig{Na} (or \chemfig{K})
  \chemar{->}
  \chemfig{R-O^{-}|^{+}Na} + $\frac{1}{2}$\chemfig{H_2}$\uparrow$
}

\chemeq{al2axd_2}{
  \chemfig{R-OH} + \chemfig{NaH}
  \hspace{8mm}\chemar{->}
  \chemfig{R-O^{-}|^{+}Na} + \chemfig{H_2}$\uparrow$
}

\subsubsection{Williamson ether synthesis \normalfont(Sections 11-14 and 14-5)}

\chemeq[al2est_2]{al2est_1}{
  \chemfig{R-O|^{-}} + \chemfig{R|'-X}
  \chemar{->}
  \chemfig{R-O-R|'} + \chemfig{X^{-}}
}
(\chemfig{R|'} must be unhindered, usually primary)



\section{Syntheses of Ethers and Epoxides}
{\large ETHERS}


\subsection{Williamson ether synthesis \normalfont(Sections 11-14 and 14-5)}

\chemeq[al2est_1]{al2est_2}{
  \chemfig{R-O|^{-}} + \chemfig{R|'-X}
  \chemar{->}
  \chemfig{R-O-R|'} + \chemfig{X^{-}}
}
(X = Cl, Br, I, OTs, etc., \chemfig{R|'} must be primary)


\subsection{Addition of an alcohol across a double bond: alkoxymercuration-demercuration \normalfont(Sections 8-6 and 14-6)}

\chemeq[kenal2est_1]{kenal2est_2}{
  \chemfig{!\Alkene} + \chemfig{Hg{(}OAc{)}_2}
  \chemar{->[\chemfig{R-OH}]}
  \chemfig{C([2]-)([4]-)([6]-[,,,2]RO)-C([0]-)([2]-)([6]-HgOAc)}
  \chemar{->[\chemfig{NaBH_4}]}
  \chemname{\hspace{1mm}\chemfig{C([2]-)([4]-)([6]-[,,,2]RO)-C([0]-)([2]-)([6]-H)}\hspace{1mm}}
           {Markovnikov orientation}
}


\subsection{Bimolecular dehydration of alcohols: industrial synthesis \normalfont(Sections 11-10B and 14-7)}

\chemeq[al2eth_1]{al2eth_2}{
  2\chemname{\chemfig{R-OH}}{primary}
  \chemar{<=>[\chemfig{H^+}]}
  \chemfig{R-O-R} + \chemfig{\Water}
}


{\large EPOXIDES}

\subsection{Peroxyacid epoxidation \normalfont(Sections 8-12 and 14-11A)}

\chemeq[ken2epx_1]{ken2epx_2}{
  \chemfig{!\Alkene} + \chemfig{R-C([2]=O)-OOH}
  \chemar{->}
  \chemfig{\vpha-C([6]-)([:60]-O?)-C?([6]-)-} + \chemfig{R-C([2]=O)-OH}
}


\subsection{Base-promoted cyclization of halohydrins \normalfont(Section 14-11B)}

\chemeq{hdrn2epx_1}{
  \chemfig{\vpha-C([2]-)([6]-OH)-C([2]-X)([6]-)-}
  \chemar{->[base]}
  \chemfig{\vpha-C([6]-)([:60]-O?)-C?([6]-)-}
  (X = Cl, Br, I, OTs, etc.)
}




\section{Reactions of Ethers and Epoxides}
{\large ETHERS}

\subsection{Cleavage by HBr and HI \normalfont(Section 14-8)}

\chemeq{eth2ah_1}{
  \chemfig{R-[,1.123]O-R|'}
  \chemar[1.5]{->[excess \chemfig{HX}][X = Br, I]}
  \chemfig{R-X} + \chemfig{R|'-X}
}

\chemeq{eth2ah_2}{
  \chemfig{Ar-O-R|'}
  \chemar[1.5]{->[excess \chemfig{HX}][X = Br, I]}
  \chemfig{Ar-OH} + \chemfig{R|'-X}
  \hspace{10mm} Ar = aromatic ring
}


\subsection{Autoxidation \normalfont(Section 14-9)}

\chemeq{eth2pxde_1}{
  \chemfig{R-O-CH_2-R|'}
  \chemar[1.2]{->[excess \chemfig{O_2}][(slow)]}
  \chemname{\chemfig{R-O-CH([2]-[,,1]OOH)-R|'}}{hydroperoxide}
  \hspace{5mm} + \hspace{5mm}
  \chemname{\chemfig{R-O-O-CH_2-R|'}}{dialkyl peroxide}
}

{\large EPOXIDES}

\subsection{Acid-catalyzed opening \normalfont(Sections 8-13 and 14-12)}

\subsubsection{In water}

\chemeq[pox2al_1]{pox2al_2}{
  \chemfig{\vpha-C([6]-)([:60]-O?)-C?([6]-)-}
  \chemar{->[\chemfig{H^+}][\chemfig{R-OH}]}
  \chemfig{\vpha-C([2]-)([6]-OH)-C([2]-OR)([6]-)-}
  \qquad \makecell{R can be hydrogen\\anti-stereochemistry}
}
The alkoxy group bonds to the more highly substituted carbon.

\subsubsection{Using hydrohalic acids (X = Cl, Br, I)}

\chemeq{epx2ah_1}{
  \chemfig{\vpha-C([6]-)([:60]-O?)-C?([6]-)-}
  \chemar{->[\chemfig{H-X}]}
  \chemfig{\vpha-C([2]-)([6]-X)-C([2]-OH)([6]-)-}
  \chemar{->[\chemfig{H-X}]}
  \chemfig{\vpha-C([2]-)([6]-X)-C([2]-X)([6]-)-}
}


\subsection{Base-catalyzed opening}

\subsubsection{With alkoxides or hydroxide \normalfont(Section 14-13)}

\chemeq{epx2eth_1}{
  \chemfig{\vpha-C([6]-)([:60]-O?)-C?H_2}
  \chemar[1.3]{->[\chemfig{R-O|^{-}}][\chemfig{R-OH}]}
  \chemfig{\vpha-C([2]-)([6]-OH)-CH_2-OR}
}
The alkoxy group bonds to the less highly substituted carbon.

\subsubsection{With organometallics \normalfont(Section 14-15)}

\chemeq{epx2eth_2}{
  \chemfig{\vpha-C([6]-)([:60]-O?)-C?H_2}
  \chemar[1.3]{->[(1)\chemfig{R-M}][(2)\chemfig{H_3O^{+}}]}
  \chemfig{\vpha-C([2]-)([6]-OH)-CH_2-OR}
  \quad M = Li or MgX
}
R bonds to the less substituted carbon.




\section{Reactions of Aromatic Compounds}


\subsection{Electrophilic aromatic substitution}

\subsubsection{Halogenation \normalfont(Section 17-2)}

\chemeq{bz2arx_1}{
  \chemfig{!\precyc**6(------)} + \chemfig{X_2}
  \chemar[2.2]{->[\chemfig{FeBr_3} for X=Br][\chemfig{AlCl_3} for X=Cl]}
  \chemfig{!\precyc**6(---(-X)---)}
  + \chemfig{HX} (X is Br or Cl)
}

\subsubsection{Nitration \normalfont(Section 17-3)}

\chemeq{bz2nbz_1}{
  \chemfig{!\precyc**6(------)} + \chemfig{HNO_3}
  \chemar{->[\chemfig{H_2SO_4}]}
  \chemname{\chemfig{!\precyc**6(---(-N([:60]=O)([:-60]-O|\rlap{$^{-}$}))---)}} {nitrobenzene}
  + \chemfig{\Water}
}
Nitration followed by reduction gives anilines.

\subsubsection{Sulfonation \normalfont (Section 17-4)}

\chemeq{bz2bzsf_1}{
  \chemfig{!\precyc**6(------)} + \chemfig{SO_3}
  \chemar[1.5]{<=>[\chemfig{H_2SO_4}][\chemfig{H_3O^+}, heat]}
  \chemname{\chemfig{!\precyc**6(---(-S([2]=O)([6]=O)-OH)---)}} {benzenesulfonic acid}
}

\subsubsection{Friedel-Crafts alkylation \normalfont(Section 17-10)}

\chemeq{bz2arr_1}{
  \chemfig{!\precyc**6(------)} + \chemfig{R-X}
  \chemar[2]{->[Lewis acid][(\chemfig{AlCl_3}, \chemfig{FeBr_3}, etc.)]}
  \chemfig{!\precyc**6(---(-R)---)} + \chemfig{H-X}
  \hspace{10mm} (X = Cl, Br, I)
}

\subsubsection{Friedel-Crafts acylation \normalfont(Section 17-11)}

\chemeq[ph2ket_2]{ph2ket_1}{
  \chemfig{!\precyc**6(------)}
  + \chemname{\chemfig{R-C([2]=O)-Cl}} {acyl halide}
  \chemar{->[\chemfig{AlCl_3}]}
  \chemfig{!\precyc**6(---(-C([:60]=O)([:-60]-R))---)}
  + \chemfig{HCl}
}

\subsubsection{Gatterman-Koch synthesis \normalfont(Section 17-11C)}

\chemeq[ph2ald_2]{ph2ald_1}{
  \chemfig{!\precyc**6(------)} + \chemfig{CO} + \chemfig{HCl}
  \chemar[1.5]{->[\chemfig{AlCl_3}/\chemfig{CuCl}]}
  \chemname{\chemfig{!\precyc**6(---(-C([:60]=O)([:-60]-H))---)}} {benzaldehyde}
}


\subsubsection{Substituent effects \normalfont(Sections 17-5 through 17-9)}

\noindent
\begin{tabular}{|c|c|c||c|c|}

  \hline

  $\pi$ donors & $\sigma$ donors & Halogens & Carbonyls & Other \\

  \hline

  \noindent
  \begin{tabular}{l}
  \chemfig{-O^{-}} \\ \\
  \chemfig{-N([2]-R)-R|'} \\ \\
  \chemfig{-OH} \\ \\
  \chemfig{-OR} \\ \\
  \chemfig{-N([2]-R)-C([2]=O)-R}
  \end{tabular}

  &

  \noindent
  \begin{tabular}{c}
  \chemnameinit{}
  \chemname{\chemfig{-R}}{alkyl} \\
  \\
  \chemname{\chemfig{-**6(------)}}{aryl} \\ (weak $\pi$ donors)
  \end{tabular}

  &

  \noindent
  \begin{tabular}{l}
  \chemfig{-F} \\ \\
  \chemfig{-Cl} \\ \\
  \chemfig{-Br} \\ \\
  \chemfig{-I}
  \end{tabular}

  &

  \noindent
  \begin{tabular}{l}
  \chemfig{-C([2]=O)-R} \\ \\
  \chemfig{-C([2]=O)-OH} \\ \\
  \chemfig{-C([2]=O)-OR}
  \end{tabular}

  &

  \noindent
  \begin{tabular}{l}
  \chemfig{-SO_3H} \\ \\
  \chemfig{-C~N} \\ \\
  \chemfig{-NO_2} \\ \\
  \chemfig{-\chemabove{N}{+}R_3}
  \end{tabular}

  \\ & & & &  \\

  \hline
  \multicolumn{3}{|c||}{ortho-, para-directing} & \multicolumn{2}{c|}{meta-directing} \\
  \hline
  \multicolumn{2}{|c|}{ACTIVATING} & \multicolumn{3}{c|}{DEACTIVATING}

  \\
  \hline

\end{tabular}
\vskip 2ex

\subsection{Nucleophilic aromatic substitution \normalfont(Section 17-12)}

\chemeq{bz2bz_1}{
  \chemname{\chemfig{**6((-G)--(-G)-(-X)---)} }
           {a halobenzene}
  + \chemnameinit{}
  \chemname{\chemfig{\Lewis{0:,Nuc}^{-}}}{strong\\nucleophile}
  \chemar{->}
  \chemfig{**6((-G)--(-G)-(-Nuc)---)} + \chemfig{X^{-}}
}
G = \chemfig{NO_2} or other strong withdrawing group. If G is not a strong electron-withdrawing group,
severe conditions are required, and a benzyne mechanism is involved (Section 17-12B).


\subsection{Addition reactions}

\subsubsection{Chlorination \normalfont(Section 17-13A)}

\chemeq{bz2bzh_1}{
  \chemfig{!\precyc**6(------)} + 3\chemfig{Cl_2}
  \chemar[2]{->[heat and pressure][or light]}
  \chemname{\chemfig{!\precyc*6( ([::-160]-H)([::160]-Cl)-
                                 ([::-40]-H)([::-80]-Cl)-
                                 ([::-40]-H)([::-80]-Cl)-
                                 ([::-40]-H)([::-80]-Cl)-
                                 ([::-40]-H)([::-80]-Cl)-
                                 ([::-40]-H)([::-80]-Cl)- ) }}
           {benzene hexachloride}
}

\subsubsection{Catalytic hydrogenation \normalfont(Section 17-13B)}

\chemeq{bz2kan_1}{
  \chemfig{**6(--(-CH_2CH_3)-(-CH_2CH_3)---)}
  \arrow{0}[,0] + \chemfig{3H_2}
  \arrow{->[Ru or Rh catalyst][100\degree C, 1000psi]}[,2]
  \chemfig{*6(--(-[:0]H)(-[:-60]CH_2CH_3)-(-[:60]H)(-[:0]CH_2CH_3)---)}
}

\subsubsection{Birch reduction \normalfont(Section 17-13C)}

\chemeq{bz2ken_1}{
  \chemfig{!\precyc**6(---(-CH_2CH_3)---)}
  \chemar[2]{->[Na or Li ][\chemfig{NH_3}$(l)$, \chemfig{R-OH}]}
  \chemfig{!\precyc*6(-=-(-CH_2CH_3)-=-)}
}
Takes place on carbon atoms bearing electron-withdrawing substituents and not on carbon atoms bearing electron-releasing substituents.


\subsection{Side-chain reactions}

\subsubsection{The Clemmensen reduction \normalfont(Section 17-11B)}

\chemeq{bz2arr_2}{
  \chemfig{[:-30]**6(---(-([:60]=O)([:-60]-R))---)}
  \chemar[2]{->[\chemfig{Zn{(}Hg{)}}][dilute \chemfig{HCl}]}
  \chemfig{[:-30]**6(---(-CH_2-R)---)}
}

\subsubsection{Permanganate oxidation \normalfont(Section 17-14A)}

\chemeq[bzkyl2bzacid_2]{bzkyl2bzacid_1}{
  \chemname{\chemfig{[:-30]**6(---(-CH_2-R)---)}}
           {an alkylbenzene}
  \arrow(.mid east--.mid west){->[hot, concd. \chemfig{KMnO_4}][\chemfig{\Water}]}[,2.2]
  \chemname{\chemfig{[:-30]**6(---(-COO^{-}|^{+}K)---)}}
           {a benzoic acid salt}
}

\subsubsection{Side-chain halogenation \normalfont(Section 17-14B)}

\chemeq{arr2arr_1}{
  \chemname{\chemfig{!\precyc**6(---(-CH_2-R)---)}}
           {an alkylbenzene}
  \arrow(.mid east--.mid west){->[\chemfig{Br_2}][$h\nu$]}
  \chemname{\chemfig{!\precyc**6(---(-([:60]-Br)([:-60]-R))---)}}
           {an $\alpha$-bromo alkylbenzene}
}

\subsubsection{Nucleophilic substitution at the benzylic position \normalfont(Section 17-14C)}

The benzylic position is activated toward both S\textsubscript N1 and S\textsubscript N2 displacements.
\chemeq{bz2bz_2}{
  \chemname{\chemfig{!\precyc**6(---(-([:60]-X)([:-60]-R))---)}}
           {an $\alpha$-halo alkylbenzene}
  + \chemfig{\Lewis{0:,Nuc}^{-}}
  \chemar{->}
  \chemfig{!\precyc**6(---(-([:60]-Nuc)([:-60]-R))---)}
  + \chemfig{X^{-}}
}


\subsection{Oxidation of phenols to quinones \normalfont(Section 17-15A)}

\chemeq{ph2qui_1}{
  \chemfig{!\precyc**6(---(=OH)-(-Cl)--)}
  \chemar[1.3]{->[\chemfig{Na_2Cr_2O_7}][\chemfig{H_2SO_4}]}
  \chemfig{O=*6(-=-(=O)-(-Cl)=-)}
}




\section{Syntheses of Ketones and Aldehydes}


\subsection{Oxidation of alcohols \normalfont(Section 11-2)}

\subsubsection{Secondary alcohols $\longrightarrow$ ketones}

\chemeq[al2ket_1]{al2ket_2}{
  \chemfig{H-C([2]-R\rlap{$'$})([6]-R\rlap{$''$})-OH}
  \chemar[1.5]{->[\chemfig{Na_2Cr_2O_7}][\chemfig{H_2SO_4}]}
  \chemfig{C([:120]-R\rlap{$'$})([:-120]-R\rlap{$''$})=O}
}

\subsubsection{Primary alcohols $\longrightarrow$ aldehydes \normalfont(Section 11-2B)}

\chemeq[al2ald_1]{al2ald_2}{
  \chemfig{R-[:30]-[:-30]OH}
  \chemar{->[\em{PCC}]}
  \chemfig{C([:120]-R)([:-120]-H)=O}
}


\subsection{Ozonolysis of alkenes \normalfont(Section 8-15B)}

\chemeq[ken2ahket_1]{ken2ahket_2}{
  \chemname{\chemfig{C([:120]-R)([:-120]-H)=C([:-60]-R|')([:60]-R|'')}}
           {alkene}
  \chemar[1.5]{->[(1) \chemfig{O_3}][(2) \chemfig{{(}CH_3{)}_2S}]}
  \chemname{\chemfig{C([:120]-R)([:-120]-H)=O}}
         {aldehyde} +
  \chemname{\chemfig{O=C([:-60]-R|')([:60]-R|'')}}
         {ketone}
}
(gives aldehydes or ketones, depending on the starting alkene)


\subsection{Friedel-Crafts acylation \normalfont(Section 17-11)}

\chemeq[ph2ket_1]{ph2ket_2}{
  \chemname{\chemfig{R-C([2]=O)-Cl}} {acyl halide}
  + \chemfig{G-**6(------)}
  \chemar{->[\chemfig{AlCl_3}]}
  \chemname{\chemfig{G-**6(---(-C([2]=O)-R)---)}} {aryl ketone}
  + (+ortho)
}
R = alkyl or aryl; G =  hydrogen, an activating group or halogen.

\subsubsection{The Gatterman-Kochformylation \normalfont(Section 17-11C)}

\chemeq[ph2ald_1]{ph2ald_2}{
  \chemfig{G-**6(------)} + \chemfig{CO} + \chemfig{HCl}
  \chemar[1.5]{->[\chemfig{AlCl_3}/\chemfig{CuCl}]}
  \chemname{\chemfig{G-**6(---(-C([:60]=O)([:-60]-H))---)}} {benzaldehyde}
}


\subsection{Hydration of alkynes \normalfont(Section 9-9F)}

\subsubsection{Catalyzed by acid and mercuric salts (Markovnikov orientation)}

\chemeq[kyn2ket_1]{kyn2ket_2}{
  \chemfig{R-C~C-H}
  \chemar[2]{->[\chemfig{Hg^{2+}}, \chemfig{H_2SO_4}][\chemfig{\Water}]}
  \chemname{\chemleft[
            \chemfig{C([:120]-R)([:-120]-[,,,2]HO)=C([:-60]-H)([:60]-H)}
            \chemright]}
           {enol (not isolated)}
  \chemar{->}
  \chemname{\chemfig{C(-[:120]R)(=[:-120]O)-CH_3}} {methyl ketone}
}

\subsubsection{Hydroboration-oxidation (anti-Markovnikov orientation)}

\chemeq[kyn2ald_1]{kyn2ald_2}{
  \chemfig{R-C~C-H}
  \chemar[2]{->[(1) \chemfig{Sia_2BH\cdot THF}][(2) \chemfig{H_2O_2}, \chemfig{NaOH}]}
  \chemname{\chemleft[
            \chemfig{C([:120]-R)([:-120]-H)=C([:-60]-H)([:60]-OH)}
            \chemright]}
           {vinyl alcohol\\(unstable)}
  \chemar{<->>}
  \chemname{\chemfig{R-C([2]-H)([6]-H)-C([:60]=O)([:-60]-H)}}
           {aldehyde\\(stable)}
}


\subsection{Alkylation of 1,3-dithianes \normalfont(Section 18-8)}

\chemeq{dith2ket_1}{
  \chemname { \chemfig{S*6(-([:-30]-H)([:-150]-H)-S----)} }
            {1,3-dithiane}
  \arrow{->[(1) \chemfig{BuLi}][(2) 1\degree \chemfig{R-X}][10pt]}[,1.5]
  \chemname{ \chemfig{S*6(-([:-30]-H)([:-150]-R)-S----)} }
           {thioacetal}
  \arrow(bb--cc){->[(1) \chemfig{BuLi}][(2) 1\degree \chemfig{R'-X}][10pt]}[,1.5]
  \chemname{ \chemfig{S*6(-([:-30]-R|')([:-150]-R)-S----)} }
           {thioketal}
  \chemnameinit{}
  \arrow(@bb--){->[*{0}\chemfig{H_3O^+}][*{0}\chemfig{HgCl_2}]}[-90]
  \chemname{ \chemfig{([2]=O)([:-150]-R)([:-30]-H)} } {aldehyde}
  \arrow(@cc--){->[*{0}\chemfig{H_3O^+}][*{0}\chemfig{HgCl_2}]}[-90]
  \chemname{ \chemfig{([2]=O)([:-150]-R)([:-30]-R|')} } {ketone}
}


\subsection{Synthesis of ketones using organolithium reagents with carboxylic acids \normalfont(Section 18-9)}

\chemeq[acid2ket_2]{acid2ket_1}{
  \chemfig{C([:120]-R)([:-120]-HO)=O}
  \chemar{->[2\chemfig{R|'-Li}]}
  \chemfig{R-C([2]-OLi)([6]-OLi)-R|'}
  \chemar{->[\chemfig{H_3O^+}]}
  \chemfig{C([:120]-R)([:-120]-R')=O}
}


\subsection{Synthesis of ketones from nitriles \normalfont(Section 18-10)}

\chemeq{ntrl2ket_1}{
  \chemfig{R-C~\Lewis{0:,N}} + \chemname{\chemfig{R|'-MgX}}{or \chemfig{R|'-Li}}
  \chemar{->}
  \chemname{\chemfig{R-C([2]=N([0]-MgX))-R|'}} {Mg salt or imine}
  \chemar{->[\chemfig{H_3O^+}]}
  \chemfig{R-C([2]=O)-R|'}
}


\subsection{Aldehyde synthesis by reduction of acid chlorides \normalfont(Section 18-11)}

\chemeq[ach2ald_2]{ach2ald_1}{
  \chemfig{R-C([2]=O)-Cl}
  \chemar[2.5]{->[\chemfig{Li^{+}|^{-}AlH{(}O\-t\-Bu{)}_3}][(or \chemfig{H_2} Pd, \chemfig{BaSO_4}, S)]}
  \chemfig{R-C([2]=O)-H}
}


\subsection{Ketone synthesis from acid chlorides \normalfont(Section 18-11)}

\chemeq{acid2ket_3}{
  \chemfig{R|'-C([2]=O)-Cl} + \chemfig{R_2CuLi}
  \chemar{->}
  \chemfig{R|'-C([2]=O)-R}
}




\section{Reactions of Ketones and Aldehydes}


\subsection{Addition of organometallic reagents \normalfont(Sections 9-7B and 10-9)}

\chemeq[ket2al_1]{ket2al_2}{
  \chemfig{R-C([2]=O)-R|'} + \chemname{\chemfig{R|''-M}} {(M=MgX, Li, etc.)}
  \chemar{->}
  \chemname{\chemfig{R-C([2]-O|^{-}|^{+}M)([6]-R|'')-R|'}}{alkoxide}
  \chemar{->[\chemfig{H_3O^+}]}
  \chemname{\chemfig{R-C([2]-OH)([6]-R|'')-R|'}}{alcohol}
}


\subsection{Reduction \normalfont(Sections 10-12 and 18-21)}

\chemeq[ket2al_3,ket2al_4]{ket2al_5}{
  \chemname{\chemfig{R-C([2]=O)-R|'}}{ketone or \\ aldehyde}
  \chemar[2.3]{->[\chemfig{NaBH_4} or \chemfig{LiAlH_4}][or \chemfig{H_2}/Ranney nickel]}
  \chemname{\chemfig{R-C([2]-O|^{-})([6]-H)-R|'}}{alkoxide}
  \chemar[0.8]{->[\chemfig{H^+}]}
  \chemname{\chemfig{R-C([2]-OH)([6]-H)-R|'}}{alcohol}
}

\subsubsection{Clemmensen reduction \normalfont(Sections 17-11B and 18-21C)}

\chemeq{ket2kan_1}{
  \chemname{\chemfig{R-C([2]=O)-R|'}}
           {ketone or aldehyde}
  + \chemfig{Zn{(}Hg{)}}
  \chemar{->[\chemfig{HCl}]}
  \chemfig{R-[:30]-[:-30]R|'}
}

\subsubsection{Wolff-Kishner reduction \normalfont(Section 18-21C)}

\chemeq{ket2kan_2}{
  \chemfig{R-C([2]=O)-R|'} +
  \chemname{\chemfig{H_2N-NH_2}}{hydrazine}
  \chemar{->}
  \chemname{\chemfig{R-C([2]=N([0]-NH_2))-R|'}} {hydrazone}
  \chemar{->[\chemfig{KOH}][heat]}
  \chemfig{R-[:30]-[:-30]R|'}
  + \chemfig{\Water} + \chemfig{N_2}$\uparrow$
}


\subsection{The Wittig reaction \normalfont(Section 18-13)}

\chemeq{ket2kan_3}{
  \chemfig{Ph_3|\chemabove{P}{\scriptstyle +}-\Lewis{0:,C}|^{-}([:60]-[,,1]R)([:-60]-[,,1]R)} +
  \chemfig{C([:120]-[,,,1]R|')([:-120]-[,,,1]R|')=O}
  \chemar{->}
  \chemfig{C([:120]-[,,,1]R)([:-120]-R)-C([:60]-R|')([:-60]-R|')}
  + \chemfig{Ph_3P=O}
}


\subsection{Hydration \normalfont(Section 18-14)}

\chemeq{ket2al_6}{
  \chemname{ \chemfig{R-C([2]=O)-R|'} } {ketone or aldehyde} + \chemfig{\Water}
  \chemar{<=>}
  \chemname{\chemfig{R-C([1]-OH)([3]-[,,,2]HO)-R|'}} {hydrate}
}


\subsection{Formation of cyanohydrins \normalfont(Section 18-15)}

\chemeq{ket2al_7}{
  \chemname{ \chemfig{R-C([2]=O)-R|'} } {ketone or aldehyde} + \chemfig{HCN}
  \chemar{<=>[\chemfig{^{-}CN}]}
  \chemname{\chemfig{R-C([1]-CN)([3]-[,,,2]HO)-R|'}} {cyanohydrin}
}


\subsection{Condensations of Amines with Ketones and Aldehydes \normalfont(Sections 18-16, 18-17, 19.10)}

\chemeq[amn2amn_2,amn2amn_3,amn2amn_4]{amn2amn_1}{
  \chemfig{!\no([:120]-)([:-120]-)=O} + \chemfig{H_2\lewis{2:,N}-Z}
  \chemar{->[\chemfig{H^+}]}
  \chemfig{!\no([:120]-)([:-120]-)=\lewis{2:,N}-Z} + \chemfig{\Water}
}

\noindent
\begin{tabular}{lllll}

Z in \chemfig{H_2\lewis{2:,N}-Z} &  \multicolumn{2}{c}{Reagent} & \multicolumn{2}{c}{Product} \\
\hline

\smallskip
\chemfig{!\no-H} &  \chemfig{H_2\lewis{2:,N}-H} & ammonia &
\chemfig{!\no([:160]-)([:-160]-)=\lewis{2:,N}-H} & an imime \\

\smallskip
\chemfig{!\no-R} &  \chemfig{H_2\lewis{2:,N}-R} & primary amine &
\chemfig{!\no([:160]-)([:-160]-)=\lewis{2:,N}-R} &  an imine (Schiff base) \\

\smallskip
\chemfig{!\no-OH} &  \chemfig{H_2\lewis{2:,N}-OH} & hydroxylamine &
\chemfig{!\no([:160]-)([:-160]-)=\lewis{2:,N}-OH} &  an oxime \\

\smallskip
\chemfig{!\no-NH_2} &  \chemfig{H_2\lewis{2:,N}-NH_2} & hydrazine &
\chemfig{!\no([:160]-)([:-160]-)=\lewis{2:,N}-NH_2} &  a hydrazone \\

\smallskip
\chemfig{!\no-NHPh} &  \chemfig{H_2\lewis{2:,N}-NHPh} & phenylhydrazine &
\chemfig{!\no([:160]-)([:-160]-)=\lewis{2:,N}-NHPh} &  a phenylhydrazone \\

\smallskip
\chemfig{!\no-NH-C([2]=O)-NH_2} & %\chemfig{H_2\lewis{2:,N}-NH-C([2]=O)-NH_2}
& semicarbazide &
%\chemfig{!\no([:160]-)([:-160]-)=\lewis{2:,N}-NH-C([2]=O)-NH_2}
&  a semicarbazone

\end{tabular}
\vskip 2ex

\subsection{Formation of acetals \normalfont(Section 18-18)}
\chemeq{ket2acl_1}{
  \chemname{\chemfig{R-C([2]=O)-R|'} }
           {ketone or aldehyde}
  \quad + \quad
  \chemname{\chemfig{R|''-OH}}{1\degree alcohol}
  \chemar{<=>[\chemfig{H^+}]}
  \chemname{\chemfig{R-C([1]-OR|'')([3]-[,,,3]R|''O)-R|'}} {acetal}
  + \chemfig{\Water}
}


\subsection{Oxidation of aldehydes \normalfont(Section 18-20)}

\chemeq{ald2acid_1}{
  \chemname{\chemfig{R-C([2]=O)-H} } {aldehyde}
  \chemar[4]{->[chromic acid, permanganate, \chemfig{Ag^+}, etc.]}
  \chemname{ \chemfig{R-C([2]=O)-OH} } {acid}
}


\subsection{Reactions of ketones and aldehydes at their $\alpha$ positions}

This large group of reactions is covered in Chapter 22.



\section{Reactions of Amines}


\subsection{Reaction as a proton base \normalfont(Section 19-5)}

\chemeq{amn2amns_1}{
  \chemname{\chemfig{R-\lewis{0:,N}([:60]-H)([:-60]-H)} } {base}
  + \chemname{\chemfig{H-X} } {proton acid}
  \chemar{<=>}
  \chemname{\chemfig{R-\chemabove{N}{\quad\scriptstyle+}(-[2]H)(-[6]H)-H|\quad X^{-} } }
           {ammonium salt}
}


\subsection{Reactions with ketones and aldehydes \normalfont(Sections 18-16, 18-17, and 19-10)}

\chemeq{amn2amm_1}{
  \chemname{\chemfig{C([2]=O)([:-30]-R|')([:-150]-R)}}
           {ketone or aldehyde} +
  \chemfig{Y-\lewis{2:,N}H_2}
  \chemar{<=>[\chemfig{H^+}]}
  \chemname{\chemleft[
            \chemfig{C([1]-\lewis{4:,N}([0]-H)([2]-Y))([3]-[,,,2]HO)([5]-R)([7]-R|')}
            \chemright]}
            {carbinolamine}
  \chemar{<=>[\chemfig{H^+}]}
  \chemname{\chemfig{C([2]=\lewis{3:,N}([1]-Y))([:-30]-R|')([:-150]-R)} }
           {derivative}
}
Y = H or alkyl gives an imine,  Y = OH gives an oxime and Y = NHR gives a hydrazone.


\subsection{Alkylation \normalfont(Section 19-12)}

\chemeq{amn2aamn_1}{
  \chemname{\chemfig{R-\lewis{2:,N}H_2}}{amine} +
  \chemname{\chemfig{R|'-CH_2-Br}}{1\degree halide}
  \chemar{->}
  \chemname{\chemfig{R-\chemabove{N}{\scriptstyle +}H_2-CH_2-R|'\quad^{-}Br}}
           {salt of alkilated amine}
}
Overalkylation is common.


\subsection{Acylation to form amides \normalfont(Section 19-13)}

\chemeq{amn2amd_1}{
  \chemname{\chemfig{Y-\lewis{2:,N}H_2}} {amine} +
  \chemname{\chemfig{R-C([2]=O)-Cl}} {acid chloride}
  \chemar{->[pyridine]}
  \chemname{ \chemfig{R-C([2]=O)-\lewis{2:,N}H-R|'} } {amide}
}


\subsection{Reaction with sulfonyl chlorides \normalfont(Section 19-14)}

\chemeq{amn2sfnd_1}{
  \chemfig{Y-\lewis{2:,N}H_2} +
  \chemname{ \chemfig{Cl-S([2]=O)([6]=O)-R|'} } {sulfonyl chloride}
  \chemar{->}
  \chemname{ \chemfig{R-\lewis{2:,N}H-S([2]=O)([6]=O)-R|'} } {sulfonide} +
  \chemfig{HCl}
}

\subsection{Hofmann and Cope eliminations}

\subsubsection{Hofmann elimination \normalfont(Section 19-15)}

Conversion to quaternary ammonium hydroxide
\chemeq{amn2amnhx_1}{
  \chemfig{R-[:30]-[:-30]-[:30]\lewis{3:,N}H_2}
  \chemar{->[3\chemfig{CH_3I}]}
  \chemfig{R-[:30]-[:-30]-[:30]\chemabove{N}{\scriptstyle+}{(}CH_3{)}_3\quad^{-}I}
}

\chemeq{amn2amnhx_2}{
  \chemfig{R-[:30]-[:-30]-[:30]\lewis{3:,N}H_2}
  \chemar{->[\chemfig{Ag_2O}]}
  \chemfig{R-[:30]-[:-30]-[:30]\chemabove{N}{\scriptstyle+}{(}CH_3{)}_3\quad^{-}OH}
}


\bigskip
Elimination

\chemeq[amn2ken_1]{amn2ken_2}{
  \chemfig{R-C([2]-[@{d3},,,5]HO@{d1}^{-}|\qquad @{d2}H)([6]-H)
           -[@{d4}]C([2]-H)([6]-[@{d5},,,2]@{d6}^{+}|N|{(}CH_3{)}_3)-H}
  \chemmove[->,red]{
    \draw(d1)..controls +(-60:5mm) and +(-120:5mm)..(d2);
    \draw(d3)..controls +(0:2mm) and +(90:2mm)..(d4);
    \draw(d5)..controls +(180:2mm) and +(90:2mm)..(d6);}
  \chemar{->[heat]}
  \chemfig{C([:120]-R)([:-120]-H)=C([:-60]-H)([:60]-H)} +
  \chemfig{\lewis{4:,N}{(}CH_3{)}_3} + \chemfig{\Water}
}
Hofmann elimination usually gives the least substituted alkene.

\subsubsection{Cope elimination of a tertiary amine oxide \normalfont(Section 19-16)}

\chemeq{amn2ken_3}{
  \chemfig{R-C([2]-H)([6]-H)-C([6]-R|')([2]-\lewis{4:,N}|{(}CH_3{)}_2)-H}
  \chemar{->[peracid][or \chemfig{H_2O_2}]}
  \chemfig{R-C([2]-[@{e1}]@{e6}H)([6]-H)-[@{e2}]C([6]-H)
           ([2]-[@{e3},,,2]@{e4}^{+}|N|{(}CH_3{)}_2([2]-[,,2,2]@{e5}^{-}|O))-H}
  \chemmove[->,red]{
    \draw(e1)..controls +(0:2mm) and +(90:2mm)..(e2);
    \draw(e3)..controls +(-180:2mm) and +(-90:2mm)..(e4);
    \draw(e5)..controls +(-180:5mm) and +(90:5mm)..(e6);}
  \chemar{->[heat]}
  \chemfig{C([:120]-R)([:-120]-H)=C([:-60]-R|')([:60]-H)} +
  \chemfig{HO-\lewis{2:,N}|{(}CH_3{)}_3}
}
Cope elimination also gives the least highly substituted alkene.


\subsection{Oxidation \normalfont(Section 19-16)}

\subsubsection{Secondary amines}

\chemeq{amn2hxamn_1}{
  \chemname{\chemfig{R_2\Lewis{2:,N}-H}} {2\degree amine} + \chemfig{H_2O_2}
  \chemar{->}
  \chemname{\chemfig{R_2\Lewis{2:,N}-OH}} {a 2\degree hydroxylamine} + \chemfig{\Water}
}

\subsubsection{Tertiary amines}

\chemeq{amn2amnox_1}{
  \chemname{\chemfig{R_3\Lewis{0:,N}}}{3\degree amine}
  \hspace{5mm} + \hspace{5mm}
  \chemname{\chemfig{H_2O_2}} {(or \chemfig{ArCO_3H})}
  \chemar{->}
  \chemname{\chemfig{R_3\chemabove{N}{+}-\chemabove{O}{-}}} {3\degree amine oxide}
  \hspace{5mm} + \hspace{5mm}
  \chemname{\chemfig{\Water}} {(or \chemfig{ArCOOH})}
}


\subsection{Diazotization (Section 19-17)}

\chemeq{amn2damn_1}{
  \chemfig{R-\Lewis{2:,N}H_2}
  \chemar{->[\chemfig{NaNO_2}][\chemfig{HCl}]}
  \chemfig{R-\chemabove{N}{+}~\lewis{0:,N}|\quad Cl^{-}}
  \qquad R can be alkyl or aryl.
}


Reactions of diazonium salts(Section 19-18)

\subsubsection{Hydrolysis}

\chemeq{damn2al_1}{
  \chemfig{Ar-\chemabove{N}{+}~\lewis{0:,N}|\quad Cl^{-}}
  \chemar{->[\chemfig{H^+}, heat][\chemfig{\Water}]}
  \chemfig{Ar-OH} + \chemfig{N_2}$\uparrow$ + \chemfig{HCl}
}

\subsubsection{The Sandmeyer reaction}

\chemeq{damn2ah_1}{
  \chemfig{Ar-\chemabove{N}{+}~\lewis{0:,N}|\quad Cl^{-}}
  \chemar[2]{->[\chemfig{CuX}][X = Cl, Br, \chemfig{C~N}]}
  \chemfig{Ar-X} + \chemfig{N_2}$\uparrow$
}

\subsubsection{Replacement by fluoride or iodide}

\chemeq{damn2ah_2}{
  \chemfig{Ar-\chemabove{N}{+}~\lewis{0:,N}|\quad Cl^{-}}
  \chemar{->[\chemfig{HBF_4}]}
  \chemfig{Ar-\chemabove{N}{+}~\lewis{0:,N}|\quad BF_4^{-}}
  \chemar{->[heat]}
  \chemfig{Ar-F} + \chemfig{BF_3} + \chemfig{N_2}$\uparrow$
}

\chemeq{damn2ah_3}{
  \chemfig{Ar-\chemabove{N}{+}~\lewis{0:,N}|\quad Cl^{-}}
  \chemar{->[\chemfig{KI}]}
  \chemfig{Ar-I} + \chemfig{KCl} + \chemfig{N_2}$\uparrow$
}

\subsubsection{Reduction to hydrogen}

\chemeq{damn2kan_1}{
  \chemfig{Ar-\chemabove{N}{+}~\lewis{0:,N}|\quad Cl^{-}}
  \chemar{->[\chemfig{H_3PO_2}]}
  \chemfig{Ar-H} + \chemfig{N_2}$\uparrow$
}

\subsubsection{Diazo coupling}

\chemeq{}{
  \chemfig{Ar-\chemabove{N}{+}~\lewis{0:,N}} \hspace{5mm} + \hspace{5mm}
  \chemname{\chemfig{H-Ar|'}} {(activated)}
  \chemar{->}
  \chemname{\chemfig{Ar-\lewis{2:,N}=\lewis{2:,N}-Ar|'}} {an azo compaund}
  + \chemfig{H^+}
}



\section{Synthesis of amines}


\subsection{Reductive amination \normalfont(Section 19-19)}

\subsubsection{Primary amines}

\chemeq[amn2amn_1]{amn2amn_2}{
  \chemfig{R-C([2]=O)-R'} +
  \chemname{\chemfig{H_2N-OH}}{hydroxilamine}
  \chemar{->[\chemfig{H^+}]}
  \chemname{\chemfig{R-C([2]=\Lewis{4:,N}([0]-OH))-R'}}{oxime}
  \chemar[1.2]{->[reduction]}
  \chemname{\chemfig{R-CH([2]-\Lewis{4:,N}H_2)-R'}}{1\degree amine}
}

\subsubsection{Secondary amines}

\chemeq[amn2amn_1]{amn2amn_3}{
  \chemfig{R-C([2]=O)-R'} +
  \chemname{\chemfig{H_2N-R''}}{1\degree amine}
  \chemar{->[\chemfig{H^+}]}
  \chemname{\chemfig{R-C([2]=\Lewis{4:,N}([0]-R''))-R'}}{N-substituted imine}
  \chemar[1.2]{->[reduction]}
  \chemname{\chemfig{R-CH([2]-\Lewis{4:,N}HR'')-R'}}{2\degree amine}
}

\subsubsection{Tertiary amines}

\chemeq[amn2amn_1]{amn2amn_4}{
  \chemfig{R-C([2]=O)-R'} +
  \chemname{\chemfig{R\rlap{$^{''}$}-NH-R|\rlap{$^{''}$}}}{2\degree amine}
  \chemar{->[\chemfig{H^+}]}
  \chemname{\chemfig{R-C([2]=\chemabove{N}{+}([4]-R\rlap{$^{''}$})([0]-R\rlap{$^{''}$}))-R'}}{iminimum salt}
  \chemar[1.2]{->[reduction]}
  \chemname{\chemfig{R-CH([2]-\Lewis{2:,N}([4]-R\rlap{$^{''}$})([0]-R\rlap{$^{''}$}))-R'}}{3\degree amine}
}


\subsection{Acylation-reduction of amines \normalfont(Section 19-20)}

\chemeq{amn2amn_5}{
  \chemname{\chemfig{R-\Lewis{2:,N}H_2}}{amine} +
  \chemname{\chemfig{Cl-C([2]=O)-R'}}{acid chloride}
  \chemar[1.2]{->[acylation]}
  \chemname{\chemfig{R-\Lewis{2:,N}H-C([2]=O)-R'}}
           {amide}
  \chemar[1.2]{->[(1) \chemfig{LiAlH_4}][(2) \chemfig{\Water}]}
  \chemname{\chemfig{R-\Lewis{2:,N}H-CH_2-R'}}{alkylated amine}
}


\subsection{Alkylation of ammonia \normalfont(Section 19-21A)}

\chemeq[ah2amn_1]{ah2amn_2}{
  \chemfig{R-[:30]-[:-30]X} + \chemname{\chemfig{\Ammonia}}{excess}
  \chemar{->}
  \chemfig{R-[:30]-[:-30]\Lewis{2:,N}H_2} + \chemfig{HX}
}


\subsection{The Gabriel synthesis of primary amines \normalfont(Section 19-21A)}

\chemeq{ah2amn_3}{
  \chemfig{R-X} + \hspace{-8mm}
  \chemname{\chemfig{!\precyc*3(-[,,,,draw=none]**6(--*5(-(=O)-\Lewis{0:4:,N}|^{-\hspace{1mm}+}K-(=O)-)----))}}
           {phthalimide anion}
  \chemar{->} \hspace{-8mm}
  \chemname{\chemfig{!\precyc*3(-[,,,,draw=none]**6(--*5(-(=O)-\Lewis{0:4:,N}(-R)-(=O)-)----))}}
           {N-alkyl phthalimide}
  \chemar[1.2]{->[\chemfig{H_2NNH_2}][heat]}
  \chemnameinit{}
  \chemname{\chemfig{R-\Lewis{2:,N}H_2}}{1\degree amine}
}


\subsection{Reduction of azides \normalfont(Section 19-21B)}

\chemeq{azd2amn_1}{
  \chemname{\chemfig{R-\Lewis{2:,N}=\chemabove{N}{+}=\chemabove[1pt]{\Lewis{2:0:,N}}{\quad\scriptstyle-}}}
           {alkyl azide}
  \chemar[1.3]{->[\chemfig{LiAlH_4}][or \chemfig{H_2/Pd}]}
  \chemname{\chemfig{R-\Lewis{2:,N}H_2}}{1\degree amine}
}


\subsection{Reduction of nitriles \normalfont(Section 19-21B)}

\chemeq{ntrl2amn_1}{
  \chemname{\chemfig{R-C~\Lewis{0:,N}}}
           {nitrile}
  \chemar[1.8]{->[\chemfig{LiAlH_4}][or \chemfig{H_2}/catalys]}
  \chemname{\chemfig{R-[:30]-[:-30]\Lewis{2:,N}H_2}}
           {1\degree amine}
}


\subsection{ Reduction of nitro compounds \normalfont(Section 19-21C)}

\chemeq{ntro2amn_1}{
  \chemfig{R-NO_2}
  \chemar[2.2]{->[active metal and \chemfig{H^+}][or \chemfig{H_2}/catalys]}
  \chemfig{R-\Lewis{2:,N}H_2}
  \qquad
  \makecell{active metal = Fe, Zn, Sn\\catalyst = Ni, Pd, or Pt}
}


\subsection{The Hofmann rearrangement \normalfont(Section 19-21D)}

\chemeq{amd2amn_1}{
  \chemname{\chemfig{R-C(=[2]O)-\Lewis{2:,N}H_2}}
           {1\degree amide}
   + \chemfig{4NaOH} + \chemname{\chemfig{X_2}}{X = Cl or Br}
  \chemar{->}
  \chemfig{R-\Lewis{2:,N}H_2}
  + \chemfig{2NaX} + \chemfig{Na_2CO_3} + \chemfig{2H_2O}
}


\subsection{Nucleophilic aromatic substitution \normalfont(Section 17-12)}

\chemeq{amn2amn_6}{
  \chemfig{R-\vpha|\Lewis{2:,N}H_2}
  +
  \chemfig{Ar-X}
  \chemar{->}
  \chemfig{R-\vpha|\Lewis{2:,N}H-Ar}
  +
  \chemfig{HX}
}
(The aromatic ring should be activated toward nucleophilic attack.)



\section{Syntheses of Carboxylic Acids}


\subsection{Oxidation of primary alcohols and aldehydes \normalfont(Sections ll-2B and 18-20)}

\chemeq{al2acid_2}{
  \chemname{\chemfig{R-[:30]-[:-30]OH}}
           {1\degree alcohol}
  \chemar[1.2]{->[\chemfig{H_2CrO_4}][or \chemfig{KMnO_4}]}
  \chemname{\chemfig{R-C(=[2]O)-H}}{aldehyde}
  \chemar[1.2]{->[\chemfig{H_2CrO_4}][or \chemfig{KMnO_4}]}
  \chemname{\chemfig{R-C(=[2]O)-OH}}{carboxilic acid}
}


\subsection{Oxidative cleavage of alkenes and alkynes \normalfont(Sections 8-15A and 9-10)}

\chemeq[ken2ket2acid_1]{ken2ket2acid_2}{
  \chemfig{C([:120]-R)([:-120]-R)=C([:-60]-H)([:60]-R|')} +
  \chemar{->[\chemfig{KMnO_4}][concd.]}
  \chemname{\chemfig{C([:120]-R)([:-120]-R)=O}}{ketone} +
  \chemname{\chemfig{O=C([:-60]-OH)([:60]-R|')}}{acid}
}

\chemeq[kyn2acid_1]{kyn2acid_2}{
  \chemfig{R-C~C-R|'}
  \chemar[2]{->[concd. \chemfig{KMnO_4}][or (1) \chemfig{O_3}, (2) \chemfig{\Water}]}
  \chemfig{R-C([:60]-OH)([:-60]=O)} +
  \chemfig{C([:120]-OH)([:-120]=O)-R|'}
}


\subsection{Oxidation of alkylbenzenes \normalfont(Section  17-14A)}

\chemeq[bzkyl2bzacid_1]{bzkyl2bzacid_2}{
  \chemname{\chemfig{**6(-(-[::120,,,,,draw=none]-[:-120,1.5]Z)--(-R|{(}alkyl{)})---)}}
           {alkylbenzene}
  \chemar[2]{->[\chemfig{Na_2Cr_2O_7},\chemfig{H_2SO_4}][or \chemfig{KMnO_4},\chemfig{!\Water}]}
  \chemname{\chemfig{**6(-(-[::120,,,,,draw=none]-[:-120,1.5]Z)--(-COOH)---)}}
           {benzoic acid}
  \makecell{Z must be \\oxidation-resistant}
}


\subsection{Carboxylation of Grignard reagents \normalfont(Section 20-SB)}

\chemeq{om2acid_1}{
  \chemfig{R-MgX}
  \chemar[1.3]{->[\chemfig{O=C=O}]}
  \chemfig{R-C(=[2]O)-O|^{-+}|MgX}
  \chemar{->[\chemfig{H^+}]}
  \chemfig{R-C(=[2]O)-OH}
}
R = alkyl or aryl


\subsubsection{Hydrolysis of nitriles \normalfont(Section 20-8C)}

\chemeq{ntrl2acid_1}{
  \chemfig{R-CH_2-C~\Lewis{0:,N}}
  \chemar[1.3]{->[\chemfig{H^+} or \chemfig{HO^{-}}][\chemfig{!\Water}]}
  \chemfig{R-CH_2-C(=[2]O)-OH}
}


\subsection{The haloform reaction\normalfont(Chapter 22)}

converts methyl ketones to acids and iodoform
\chemeq[ket2acid_2]{ket2acid_1}{
  \chemfig{R-C(=[2]O)-CH_3}
  \chemar{->[\chemfig{X_2}][\chemfig{HO^{-}}]}
  \chemfig{R-C(=[2]O)-O^{-}} + \chemfig{HCX_3}
  \qquad X = Cl, Br, I
}


\subsection{Malonic ester synthesis \normalfont(Chapter 22)}

makes substituted acetic acids
\chemeq[malest_2]{malest_1}{
  \chemfig{CH_2(-[2]COOEt)(-[6]COOEt)}
  \chemar[2]{->[(1) \chemfig{Na^{+-}OCH_2CH_3}][(2) \chemfig{R-X}]}
  \chemfig{HC(-[2,,2]COOEt)(-[6,,2]COOEt)-R}
  \chemar[1.3]{->[(1)\chemfig{HO^{-}}][(2)\chemfig{H^+}, heat]}
  \chemfig{R-[:-30]-[:30]C(-[2]OH)=O} + \chemfig{CO_2}
}

\section{Reactions of Carboxylic Acids}


\subsection{Salt formation \normalfont(Section 20-5)}

\chemeq{acid2acid_1}{
  \chemfig{R-C(=[2]O)-OH} +
  \chemname{\chemfig{M^{+-}OH}}{strong base}
  \chemar{<->>}
  \chemfig{R-C(=[2]O)-\vpha|O^{-+}M} + \chemfig{!\Water}
}


\subsection{ Conversion to esters \normalfont(Sections 20-10, 20-11, and 20-15)}
Fisher esterification

\chemeq{acid2est_1}{
  \chemname{\chemfig{R-C(=[2]O)-OH}}{acid}
  +
  \chemname{\chemfig{R'-OH}}{alcohol}
  \chemar{<=>[\chemfig{H^+}]}
  \chemname{\chemfig{R-C(=[2]O)-O-R'}}{ester}
}


\subsection{Conversion to amides \normalfont(Sections 20-12 and 20-15)}

\chemeq{acid2amd_1}{
  \chemname{\chemfig{R-C(=[2]O)-OH}}{acid}
  +
  \chemname{\chemfig{H_2N-R'}}{amine}
  \chemar{<->>}
  \chemname{\chemfig{R-C(=[2]O)-O|^{-}\hspace{2mm}H_3N\rlap{$^+$}-R'}}{salt}
  \chemar{->[heat]}
  \chemname{\chemfig{R-C(=[2]O)-NH-R'}}{amide}
}


\subsection{Conversion to anhydrides \normalfont(Section 21-5)}

\chemeq{acid2ahdr_1}{
  \chemname{\chemfig{R-C(=[2]O)-OH}}{acid}
  +
  \chemname{\chemfig{Cl-C(=[2]O)-R'}}{acid chloride}
  \chemar{->}
  \chemname{\chemfig{R-C(=[2]O)-O-C(=[2]O)-R'}}{acid anhydride}
  +
  \chemfig{HCl}
}


\subsection{Reduction to primary alcohols \normalfont(Sections 10-11 and 20-13)}

\chemeq[acid0est2alco_1]{acid0est2alco_2}{
  \chemname{\chemfig{R-C(=[2]O)-OH}}{acid}
  \chemar[3]{->[(1) \chemfig{LiAlH_4}][(2) \chemfig{H_3O^+} (or use {\em BH$_3\cdot$THF})]}
  \chemname{\chemfig{R-[:30]-[:-30]OH}}{1\degree alcohol}
}


\subsection{Reduction to aldehydes \normalfont(Sections 18-11 and 20-13)}

\chemeq[ach2ald_1]{ach2ald_2}{
  \chemname{\chemfig{R-C(=[2]O)-Cl}}{acid chloride}
  \chemar[4]{->[(1) \chemfig{LiAl{(}OC{(}CH_3{)}_3{)}_3H}][lithium tri-{\em tert}-butoxyaluminum hydride]}
  \chemname{\chemfig{R-C(=[2]O)-H}}{aldehyde}
}


\subsection{Alkylation to form ketones \normalfont(Sections 18-9 and 20-14)}

\chemeq[acid2ket_1]{acid2ket_2}{
  \chemname{\chemfig{R-C(=[2]O)-O|^{-+}Li}}{lithium carboxilate}
  \qquad + \qquad
  \chemname{\chemfig{Li-R'}}{alkyllythium}
  \chemar{->[\chemfig{!\Water}]}
  \chemname{\chemfig{R-C(=[2]O)-R'}}{ketone}
}


\subsection{Conversion to acid chlorides \normalfont(Section 20-15)}

\chemeq{acid2acidh_1}{
  \chemname{\chemfig{R-C(=[2]O)-OH}}{acid}
  +
  \chemname{\chemfig{Cl-S(=[2]O)-Cl}}{thionyl chloride}
  \chemar{->}
  \chemname{\chemfig{R-C(=[2]O)-Cl}}{acid chloride}
  +
  \chemfig{SO_2\uparrow}
  +
  \chemfig{HCl\uparrow}
}


\subsection{Side-chain halogenation (Hell-Volhard-Zelinsky reaction), \normalfont(Section 22-4)}

\chemeq[hvz_2]{hvz_1}{
  \chemnameinit{\chemfig{\Chembelow{C}{H}=[2]O}}
  \chemname{\chemfig{-\chembelow{C_{\hphantom{2}}}{H_2}-C(=[2]O)-\vphantom{H_2}OH}}{acid}
  \chemar[1.2]{->[\chemfig{Br_2}/\chemfig{PBr_3}]}
  \chemname{\chemfig{-\chembelow{C}{H}(-[2]Br)-C(=[2]O)-Br}}{$\alpha$-bromo acyl bromide}
  \chemar{->[\chemfig{!\Water}]}
  \chemname{\chemfig{-\chembelow{C}{H}(-[2]Br)-C(=[2]O)-OH}}{$\alpha$-bromoacid}
  +
  \chemfig{HBr}
}



\section{Enolate Additions and Condensations}

A complete summary of additions and condensations would be long and involved. This summary covers the major classes of condensations and related reactions.

\subsection{Alkylation of lithium enolates \normalfont(Section 22-3)}

\chemeq{ket2ket_1}{
  \chemfig{R-C(=[2]O)-CH_2-R}
  \chemar[1.3]{->[(1) LDA][(2) {\setchemfig{chemfig style={green}}\chemfig{R'-X}}]}
  \chemfig{R-C(=[2]O)-CH(-[2]\grn{R}|\grn{'})-R}
}
(LDA = lithium diisopropylamide; \chemfig{R'-X} = unhindered 1\degree halide or tosylate)


\subsection{Alkylation of enamines (Stork reaction) \normalfont(Section 22-4)}

\chemeq{ah2ket_1}{
  \chemname{\chemfig{C(-[:-120])(-[@{x2}:120]@{x1}\Lewis{0:,N}(-[:60]R)(-[4]R))=[@{y1}]C(-[:60])(-[:-60])}}
           {eminamine}
  \chemar{->[{\setchemfig{chemfig style={green}}\chemfig{@{y2}R'-[@{z1}]@{z2}X}}]}
  \chemname{\chemfig{C(-[:-120])(=[:120]N(-[:120,0.4,,,draw=none]{+}-[:120,0.4,,,draw=none]X^{-})(-[:60,,1]R)(-[4]R))-C(-[6])(-[2]\grn{R}|\grn{'})-}}
           {alkylated eminamine}
  \chemar{->[\chemfig{H_3O^+}]}
  \chemname{\chemfig{C(-[:-120])(=[:120]O)-C(-[6])(-[2]\grn{R}|\grn{'})-}}
           {alkylated ketone}
  +
  \chemfig{R-N\rlap{$^+$}(-[2]H)(-[6]H)-R}
  \chemmove{\draw[->,red](x1)..controls +(0:5mm) and +(45:2mm)..(x2);}
  \chemmove{\draw[->,red](y1)..controls +(90:10mm) and +(135:10mm)..(y2);}
  \chemmove{\draw[->,red](z1)..controls +(90:3mm) and +(135:3mm)..(z2);}
}


\subsection{$\alpha$ halogenation \normalfont(Section 22-5)}

\chemeq{ket2ket_2}{
  \chemfig{R-C(=[2]O)-C\rlap{\color{red}$^\alpha$}(-[2]H)(-[6])-}
  +
  {\setchemfig{chemfig style={green}}\chemfig{X_2}}
  \chemar{->[\chemfig{H^+} or][\chemfig{HO^{-}}]}
  \chemfig{R-C(=[2]O)-C(-[2]\grn{X})(-[6])-}
}

\subsubsection{The iodoform (or haloform) reaction \normalfont(Section 22-5B)}

\chemeq[ket2acid_1]{ket2acid_2}{
  \chemname{\chemfig{R-C(=[2]O)-CH_3}}{methyl ketone}
  \quad + \quad
  \chemname{\chemfig{I_2}}{excess}
  \chemar{->[\chemfig{HO^{-}}]}
  \chemfig{R-C(=[2]O)-O^{-}}
  +
  \chemfig{HCI_3\downarrow}
}

\subsubsection{The Hell-Volhard-Zelinsky (HVZ) reaction \normalfont(Section 22-6)}

\chemeq[hvz_1]{hvz_2}{
  \chemname{\chemfig{-CH_2-C(=[2]O)-\vphantom{H_2}OH}}{acid}
  \chemar[1.2]{->[{\setchemfig{chemfig style={green}}\chemfig{Br_2}/\chemfig{PBr_3}}]}
  \chemname{\chemfig{-CH(-[2]\grn{Br})-C(=[2]O)-\grn{Br}}}
           {$\alpha$-bromo acyl bromide}
  \chemar{->[\chemfig{!\Water}]}
  \chemname{\chemfig{-CH(-[2]\grn{Br})-C(=[2]O)-OH}}{$\alpha$-bromoacid}
}


\subsection{The aldol condensation and subsequent dehydration \normalfont(Sections 22-7 through 22-11)}

\chemeq{ket2ket_3}{
  \chemname{
      \chemfig{H_2C(-[6,,2]R|')(-[2,,2]C(=[4]O)-[2]R)
      -[,,,,draw=none]
      \grn{C}(-[2,,,,green]\grn{R})(=[,,,,green]\grn{O})-[6,,,,green]\grn{C}|\grn{H_2}-[6,,,,green]\grn{R}|\grn{'}}
    }{ketone or aldehyde}
  \chemar{<=>[\chemfig{H^+} or][\chemfig{HO^{-}}]}
  \chemname{
      \chemfig{HC(-[6,,2]R|')(-[2,,2]C(=[4]O)-[2]R)-
      \grn{C}(-[2,,,,green]\grn{R})(-[,,,,green]\grn{O}\grn{H})-[6,,,,green]\grn{C}|\grn{H_2}-[6,,,,green]\grn{R}|\grn{'}}
    }{aldol product}
  \chemar[1.5]{<=>[heat][\chemfig{H^+} or \chemfig{HO^{-}}]}
  \chemname{
      \chemfig{C(-[6]R|')(-[2]C(=[4]O)-[2]R)=
      \grn{C}(-[2,,,,green]\grn{R})-[6,,,,green]\grn{C}|\grn{H_2}-[6,,,,green]\grn{R}|\grn{'}}
    }{$\alpha,\beta$-unsaturated\\ketone or aldehyde}
  +
  \chemfig{!\Water}
}


\subsection{The Claisen ester condensation \normalfont(Sections 22-12 through 22-14)
            \\(Cyclizations are the Dieckmann condensation.)}

\chemeq{est2est_1}{
  \chemfig{RO-C(=[6]O)-CH_2(-R')-[2,,,,draw=none]
    \grn{C}(-[4,,,,green]\grn{R}|\grn{O})(=[2,,,,green]\grn{O})-[,,,,green]\grn{C}|\grn{H_2}-[,,,,green]\grn{R}|\grn{'}}
  \chemar{<=>[\chemfig{RO^{-}}]}
  \chemfig{RO-C(=[6]O)-CH(-R')-[2]
      \grn{C}(=[2,,,,green]\grn{O})-[,,,,green]\grn{C}|\grn{H_2}-[,,,,green]\grn{R}|\grn{'}}
  +
  \chemfig{\grn{R}\grn{O}H}
}
The product is initially formed as its anion.


\subsection{The malonic ester synthesis \normalfont(Section 22-16)}

\chemeq[malest_1]{malest_2}{
  \chemname{\chemfig{H-C(-H)(-[2]COOEt)(-[6]COOEt)}}
           {malonic ester}
  \chemar[1.8]{->[(1) \chemfig{NaOCH_2CH_3}][(2)\setchemfig{chemfig style=green}\chemfig{R-X}]}
  \chemname{\chemfig{\grn{R}-C(-H)(-[2]COOEt)(-[6]COOEt)}}
           {substitutes\\malonic ester}
  \chemar{->[\chemfig{H_3O}][heat]}
  \chemname{\chemfig{\grn{R}-[:30]-[:-30]COOH} + \chemfig{CO_2}}
           {substituted\\acetic acid}
}






\subsection{The acetoacetic ester synthesis \normalfont(Section 22-17)}

\chemeq{est2ket_1}{
  \chemname{\chemfig{H-C(-H)(-[2]COOEt)(-[6]C(=[4]O)-CH_3)}}
           {acetoacetic ester}
  \chemar[1.8]{->[(1) \chemfig{NaOCH_2CH_3}][(2)\setchemfig{chemfig style=green}\chemfig{R-X}]}
  \chemname{\chemfig{\grn{R}-C(-H)(-[2]COOEt)(-[6]C(=[4]O)-CH_3)}}
           {substituted\\acetoacetic ester}
  \chemar{->[\chemfig{H_3O}][heat]}
  \chemname{\chemfig{\grn{R}-CH_2-[6]C(=[4]O)-CH_3} + \chemfig{CO_2}}
           {substituted acetone}
}


\subsection{The Michael addition (conjugate addition) \normalfont(Sections 22-18 and 22-19)}

\chemeq{ken2kan_2}{
  \chemfig{H-\Lewis{0:,C}|^{-}(-[2]Y)(-[6]Z)}
  +
  \setchemfig{chemfig style=green}
  \chemfig{C(-[:120])(-[:-120])=C(-[2])-C(=[2]O)-}
  \chemar[1.5]{->[ROH][(proton source)]}
  \chemfig{\blk{H}-[,,,,black]\blk{C}(-[2,,,,black]\blk{Y})(-[6,,,,black]\blk{Z})-C(-[2])(-[6])-C(-[2,,,,black]\blk{H})(-[6])-C(=[2]O)-}
}
(Y and Z are carbonyl or other electron-withdrawing groups.)






\section{Reactions of carbohydrates}

\subsection{Undesirable rearrangements catalyzed by base \normalfont(Section 23-8)}
Because of these side reactions, basic reagents are rarely used with sugars.

\subsubsection{Epimerization of the alpha carbon}

\chemeq{sug2sug_1}{
  \arrow{0}[0]
  \chemfig{[2]CH_2OH-[,,,2]!Z-[,,2]!x-CHO}
  \arrow{<=>[\chemfig{HO^{-}}]}
  \chemfig{[2]CH_2OH-[,,,2]!Z-[,,2]!y-CHO}
}

\subsubsection{Enediol rearrangements}

\newcommand{\phantomas}{\vphantom{\chemfig{[2]CH_2OH-!z-!z-C(-[:0]OH)=C(-[:150]H)-[:30]OH}}}
\chemeq{sug2sug_2}{
  \arrow{0}[,0]\phantomas
  \chemfig{[2]CH_2OH-!z-!z-!z-CHO}
  \arrow{<=>[\chemfig{HO^{-}}]}\phantomas
  \chemfig{[2]CH_2OH-!z-!z-C(-[:0]OH)=C(-[:150]H)-[:30]OH}
  \arrow{<=>[\chemfig{HO^{-}}]}\phantomas
  \chemfig{[2]CH_2OH-!z-!z-C(=[:0]O)-CH_2OH}
  \arrow{<=>[\chemfig{HO^{-}}]}\phantomas
  \chemfig{[2]CH_2OH-!z-C(-[:0]OH)=C(-[:150]HO)-[:30]CH_2OH}
  \arrow{<=>[\chemfig{HO^{-}}]}\phantomas
  \chemfig{[2]CH_2OH-!z-C(=[:0]O)-!z-CH_2OH}
}


\subsection{Reduction  \normalfont(Section 23-9)}

\chemeq{sug2sug_3}{
  \arrow{0}[0]
  \chemfig{[2]CH_2OH-[,,,2]!Z-[,,2]CHO}
  \arrow{<=>[\chemfig{NaBH_4}][or \chemfig{H_2}/\chemfig{Ni}]}[,1.4]
  \chemfig{[2]CH_2OH-[,,,2]!Z-[,,2]CH_2OH}
}


\subsection{Oxidation \normalfont(Section 23-10)}

\subsubsection{To aldonic acids (glyconic acids) by bromine water}

\chemeq{sug2acid_1}{
  \arrow{0}[0]
  \chemfig{[2]CH_2OH-[,,,2]!Z-[,,2]CHO}
  \arrow{->[\chemfig{Br_2}][\chemfig{H_2O}]}
  \chemfig{[2]CH_2OH-[,,,2]!Z-[,,2]COOH}
}

\subsubsection{To aldaric acids (glycaric acids) by nitric acid}

\chemeq{sug2acid_2}{
  \arrow{0}[0]
  \chemfig{[2]CH_2OH-[,,,2]!Z-[,,2]CHO}
  \arrow{->[\chemfig{HNO_3}]}
  \chemfig{[2]COOH-[,,,2]!Z-[,,2]COOH}
}

\subsubsection{Tollens test for reducing sugars}

\chemeq{sug2acid_3}{
  \arrow{0}[0]
  \chemfig{[2]CH_2OH-[,,,2]!Z-[,,2]!z-CHO}
  \arrow{0}[,0.25] or \arrow{0}[,0.25]
  \chemfig{[2]CH_2OH-[,,,2]!Z-[,,2]C(=[0]O)-CH_2OH}
  \arrow{->[\chemfig{Ag{(}NH_3{)}_2OH}]}[,2]
  \chemfig{[2]CH_2OH-[,,,2]!Z-[,,2]!z-COO^{-}}
  \arrow{0}[,0.25] \makecell[l]{\+ \chemfig{Ag} (silver mirror)\\ \+ rearrangement}
}

\subsection{Glycoside formation (conversion to acetal) \normalfont(Section 23-11)}

\chemeq{sug2acl_1}{
\chemfig{
  ?(-[:190]HO)
  -[:-50](-[:170]HO)
  -[:10](-[:-55,0.7,,2]HO)
  -[:-10]\charge{-5:12mm=\grn{\small\it either anomer}}{}
      (-[6,0.7,1,,green,decorate,decoration={snake,amplitude=0.3mm,segment length=0.7mm}]OH)
      (-[1,0.7,1,,green,decorate,decoration={snake,amplitude=0.3mm,segment length=0.7mm}]H)
  -[:130]O
  -[:190]
  ?(-[:150,0.7]-[2,0.7]OH) }
\hphantom{123456789}
\arrow{->[\chemfig{CH_3OH}][\chemfig{H^+}]}[,1.3]
\chemfig{
  ?(-[:190]HO)
  -[:-50](-[:170]HO)
  -[:10](-[:-55,0.7,,2]HO)
  -[:-10]\charge{-10:20mm=\makecell[l]{\color{green}\small\it$\alpha$ + $\beta$\\\color{green}\small\it more stable predominates}}{}
    (-[6,0.7,,,green,decorate,decoration={snake,amplitude=0.3mm,segment length=0.7mm}]H)
    (-[1,0.7,,,green,decorate,decoration={snake,amplitude=0.3mm,segment length=0.7mm}]OCH_3)
  -[:130]O
  -[:190]
  ?(-[:150,0.7]-[2,0.7]OH) }
  \vpha
}

\subsection{Alkylation to give ethers \normalfont(Section 23-12)}

\chemeq{sug2eth_1}{
\arrow{0}[0]
\chemfig{
  ?(-[:190]HO)
  -[:-50](-[:170]HO)
  -[:10](-[:-55,0.7,,2]HO)
  -[:-10](-[6,0.7,1]OH)
  -[:130]O
  -[:190]
  ?(-[:150,0.7]-[2,0.7]OH) }
\arrow{->[excess \chemfig{CH_3I}][\chemfig{Ag_2O}]}[,1.7]
\chemfig{
  ?(-[:190]H_3CO)
  -[:-50](-[:170]H_3CO)
  -[:10](-[:-55,0.7,,3]H_3CO)
  -[:-10](-[6,0.7]OCH_3)
  -[:130]O
  -[:190]
  ?(-[:150,0.7]-[2,0.7]OCH_3) }
\makecell{gives the same anomer\\as starting material}
}

\subsection{Acylation to give esters \normalfont(Section 23-12)}

\chemeq{sug2est_1}{
\arrow{0}[0]
\chemfig{
  ?(-[:190]HO)
  -[:-50](-[:170]HO)
  -[:10](-[:-55,0.7,,2]HO)
  -[:-10](-[6,0.7,1]OH)
  -[:130]O
  -[:190]
  ?(-[:150,0.7]-[2,0.7]OH) }
\arrow{->[excess \chemfig{Ac_2O}][pyridine]}[,1.7]
\chemfig{
  ?(-[:190]AcO)
  -[:-50](-[:170]AcO)
  -[:10](-[:-55,0.7,,2]AcO)
  -[:-10](-[6,0.7]OAc)
  -[:130]O
  -[:190]
  ?(-[:150,0.7]-[2,0.7]OAc) }
\makecell{gives the same anomer\\as starting material}
}

\subsection{Osazon formation \normalfont(Section 23-13)}

\chemeq{sug2osz_1}{
  \arrow{0}[0]
  \chemfig{[2]CH_2OH-[,,,2]!Z-[,,2]!z-CHO}
  \arrow{0}[,0.25] or \arrow{0}[,0.25]
  \chemfig{[2]CH_2OH-[,,,2]!Z-[,,2]C(=[0]O)-CH_2OH}
  \arrow{->[excess \chemleft[\chemfig{Ph-NHNH_2}\chemright]][\chemfig{H^+}]}[,2.5]
  \chemfig{[2]CH_2OH-[,,,2]!Z-[,,2]C(=[0]N-[0]NHPh)-[,,,2]HC(=[0]N-[0]NHPh)}
  \arrow{0}[,0.25] osazone
}

\subsection{Ruff degradation \normalfont(Section 23-14)}

\chemeq{sug2sug_4}{
  \arrow{0}[0]
  \chemfig{[2]CH_2OH-[,,,2]!Z-[,,2]!z-CHO}
  \arrow{->[1. \chemfig{Br_2}/\chemfig{H_2O}][2. \chemfig{H_2O_2},\chemfig{Fe_2{(}SO_4{)}_3}]}[,2.2]
  \vphantom{\chemfig{[2]CH_2OH-[,,,2]!Z-[,,2]!z-CHO}}
  \chemfig{[2]CH_2OH-[,,,2]!Z-[,,2]CHO}
  \arrow{0}[,0]
  \+
  \arrow{0}[,0]
  \chemfig{CO_2\uparrow}
}

\subsection{Kiliani-Fisher synthesis \normalfont(Section 23-14)}

\chemeq{sug2sug_5}{
  \arrow{0}[0]
  \vphantom{\chemfig{[2]CH_2OH-[,,,2]!Z-[,,2]!z-CHO}}
  \chemfig{[2]CH_2OH-[,,,2]!Z-[,,2]CHO}
  \arrow{->[1. \chemfig{HCN}/\chemfig{KCN}\hphantom{aaa}][\makecell[l]{2. \chemfig{H_2}/\chemfig{Pd{(}BaSO_4{)}} \\ 3. \chemfig{H_3O^+}}]}[,2.2]
  \chemfig{[2]CH_2OH-[,,,2]!Z-[,,2]!z-CHO}
}

\subsection{Formose (Butlerov) reaction}

\chemeq{ald2sug_1}{
  \arrow{0}[,0.1]
  \vphantom{\chemfig{(=[2]O)-[:-30]-[:-90]H}}
  {\it n}\chemfig{(-[:210]H)(=[2]O)-[:-30]H}
  \arrow{->[\chemfig{Ca{(}OH{)}_2}][\chemfig{H_2O}]}[,1.3]
  \chemfig{(-[:210]H)(=[2]O)-[@{op}:-30](-[6]OH)-[@{cl}:30]-[:-30]OH}
  \polymerdelim[indice=\!\!n-2]{op}{cl}
}



\section{Synthesis of Amino Acids}

\subsection{Reductive amination \normalfont(Section 24-5A)}

\chemeq{acid2imn2ama_1}{
\chemname{\chemfig{R-C(=[2]O)-COOH}}{$\alpha$-ketoacid}
\arrow{->[excess \chemfig{NH_3}]}[,1.6]
\chemname{\chemfig{R-C(=[2]NH)-COO^{-}}\chemfig{\vpha^{+}NH_4}}{imine}
\arrow{->[\chemfig{H_2}][\chemfig{Pd}]}
\chemname{\chemfig{R-CH(-[2]NH_2)-COO^{-}}}{$\alpha$-amino acid}
}

\subsection{Amination of an $\alpha$-haloacid \normalfont(Section 24-5B)}

\chemeq{acid2ama_1}{
\chemname{\chemfig{R-CH_2-C(=[2]O)-OH}}{carboxylic acid}
\arrow{->[1. \chemfig{Br_2}/\chemfig{PBr_3}][2. \chemfig{{H_2O}}]}[,1.6]
\chemname{\chemfig{R-CH(-[2]Br)-C(=[2]O)-OH}}{$\alpha$-bromo acid}
\arrow{->[\chemfig{NH_3}][large excess]}[,1.5]
\chemname{\chemfig{R-CH(-[2]NH_2)-COO^{-}}\chemfig{\vpha^{+}NH_4}}{(D,L)-$\alpha$-amino salt}
}


\subsection{The Gabriel malonic ester synthesis \normalfont(Section 24-5C)}

\chemeq{est2ama_1}{
  \setchemfig{atom sep=1em}
  \chemname{\chemfig{**6(--*5(-(=o)-N(-[,2]CH(-[2,2]COOEt)(-[6,2]COOEt))-(=o)-)----)}}
           {N-phthalimidomalonic ester}
  \arrow{->[1. base][2. \chemfig{R-X}]}[,1.3]
  \setchemfig{atom sep=1em}
  \chemname{\chemfig{**6(--*5(-(=o)-N(-[,2]C(-[2,2]COOEt)(-[6,2]COOEt)-[,2]R)-(=o)-)----)}}
           {alkylated}
  \arrow{->[\chemfig{H_3O^+}][heat]}
  \chemname{\chemleft[\chemfig{H_3\chemabove{N}{+}-C(-[2]COOH)(-[6]COOH)-R}\chemright]}
           {hydrolized}
  \arrow{->[heat]}
  \chemname{\chemfig{H_3\chemabove{N}{+}-C(-[2]H)(-[6]COOH)-R}}
           {$\alpha$-amino acid}
}


\subsection{The Strecker synthesis \normalfont(Section 24-5D)}

\chemeq{ald2ntrl2ama_1}{
\chemname{\chemfig{R-CH(=[2]O)}}{aldehyde}
\+ \chemfig{NH_3} \+ \chemfig{HCN}
\arrow{->[\chemfig{H_2O}]}
\chemname{\chemfig{R-CH(-[2]NH_2)-C~N}}{$\alpha$-amino nitrile}
\arrow{->[\chemfig{H_3O^+}]}
\chemname{\chemfig{R-CH(-[2,,,2]\vpha^+NH_3)-COOH}}{$\alpha$-amino acid}
}




\section{Reactions of Amino Acids}

\subsection{Esterification of the carboxyl group \normalfont(Section 24-7A)}

\chemeq{ama2est_1}{
  \chemname{\chemfig{R-CH(-[2,,,2]\vpha^+NH_3)-COOH}}
           {amino acid}
  \+
  \chemname{\chemfig{\Charge{0=$'$}{R}-OH}}{alcohol}
  \arrow{->[\chemfig{H^+}]}
  \chemname{\chemfig{R-CH(-[2,,,2]\vpha^+NH_3)-C-O-\Charge{0=$'$}{R}}}
           {amino ester}
  \+
  \chemfig{H_2O}
}


\subsection{Acylation of the amino group: formation of amides \normalfont(Section 24-7B)}

\chemeq{ama2amd_1}{
  \chemname{\chemfig{\Charge{0=$'$}{R}-C(=[2]O)-X}}{acylating agent}
  \+
  \chemname{\chemfig{NH_2-CH(-[2]R)-COOH}}
           {amino acid}
  \arrow{->}
  \chemname{\chemfig{\Charge{0=$'$}{R}-C(=[2]O)-NH-CH(-[2]R)-COOH}}
           {acylated amino acid}
  \+
  \chemfig{H-X}
}


\subsection{Reaction with ninhydrin \normalfont(Section 24-7C)}

\chemeq{ama2ald_1}{
  \chemnameinit{\chemfig[atom sep=1.4em]{!\precyc*3(2-**6(--*5(-(=O)-(-[1]OH)(-[7]OH)-(=O)-)----))}}
  \chemname{\chemfig{NH_2-CH(-[2]R)-COOH}}
           {amino acid}
  \+
  \setchemfig{atom sep=1.4em}
  \chemname{\chemfig{!\precyc*3(2-[,,,,draw=none]**6(--*5(-(=O)-(-[1]OH)(-[7]OH)-(=O)-)----))}}
           {ninhydrin}
  \arrow(.mid east--.mid west) {->[pyridine]}[,1.2]
  \setchemfig{atom sep=1.3em}
  \chemname{\chemfig{!\precyc*3(-[,,,,draw=none]**6(--*5(-(=O)-(=N-*5(-(=O)-**6(------)--(-[,,,1]O|^{-})=))-(=O)-)----))}}
           {Ruhemann's purple}
  \arrow{0}[,0.25] \makecell[l]{\+ \chemfig{R-CHO}\\ \+ \chemfig{CO_2\uparrow}}
}


\subsection{Formation of peptide bonds \normalfont(Sections 24-10 and 24-11)}

\chemeq{ama2ppt_1}{
  \chemfig{H_3\chemabove{N}{+}-CH(-[6]R_1)-C(=[2]O)-O|^{-}}
  \+
  \chemfig{H_3\chemabove{N}{+}-CH(-[6]R_2)-C(=[2]O)-O|^{-}}
  \arrow
  \chemfig{H_3\chemabove{N}{+}-CH(-[6]R_1)-C(=[2]O)-NH-CH(-[6]R_2)-C(=[2]O)-O|^{-}}
  \+
  \chemfig{H_2O}
}

















%\end{preview}

\end{document}















