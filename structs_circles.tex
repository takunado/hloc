\documentclass[a4paper,twoside,aps,prl,preprint,10pt,notitlepage]{revtex4-1}
\usepackage[active,tightpage]{preview}
\usepackage[head=12pt,includehead,inner=4cm,outer=1cm,top=1cm,bottom=1.5cm]{geometry}
\usepackage[utf8]{inputenc}
\usepackage{chemfig}
\usepackage{longtable}
\usepackage{amssymb}
\usepackage{multirow}
\usepackage[verbose]{wrapfig}
\usepackage{textpos}
\usepackage{fancyhdr}
\usepackage{makecell}
\usepackage{array} 
\usepackage{amsmath}
\usepackage{ifthen}
\usepackage{cleveref}
%\usepackage{titlesec}

\renewcommand{\PreviewBorder}{1in}
\newcommand{\Newpage}{\end{preview}\begin{preview}}


\pagestyle{fancy}
\fancyheadoffset{0pt}
\renewcommand{\sectionmark}[1]{\markboth{#1}{}}
\fancyhf{}
\fancyhead[LE,RO]{\leftmark}
\fancyhead[RE,LO]{\thepage}
\fancypagestyle{plain}{%
  \fancyhead{} % get rid of headers
  \renewcommand{\headrulewidth}{0pt} % and the line
}

%\titlespacing\section{0pt}{15pt plus 4pt minus 2pt}{10pt plus 2pt minus 2pt}
%\titlespacing\subsection{0pt}{12pt plus 4pt minus 2pt}{8pt plus 2pt minus 2pt}
%\titlespacing\subsubsection{0pt}{8pt plus 4pt minus 2pt}{5pt plus 2pt minus 2pt}
\setlength{\TPHorizModule}{\textwidth}
\setlength{\TPVertModule}{\textheight}
%\setlength{\parskip}{0cm}

\let\stdsection\section
\renewcommand\section{\newpage\stdsection}

\newcommand{\vpha}{\vphantom{A}}
\newcommand{\degree}{\ensuremath{^\circ}}
\newcommand{\chemar}[2][]{
  \schemestart[{},#1]
  \vpha\arrow{#2}\vpha
  \schemestop
}
\newcommand{\chemeq}[3][]{
  \chemnameinit{}
  \noindent
  \begin{equation}
    \schemestart[][base]
    #3
    \schemestop
    \hskip \textwidth minus \textwidth
    \ifthenelse{\equal{#1}{}}{}{\cref{#1}\Leftarrow}
    \ifthenelse{\equal{#2}{}}{}{\label{#2}}
  \end{equation}
}
\newcommand{\precyc}{\vpha-[,0.1,,,draw=none]}

\newcommand{\grn}[1]{\vpha{\color{green}#1}}
\newcommand{\blu}[1]{\vpha{\color{blue}#1}}
\newcommand{\blk}[1]{\vpha{\color{black}#1}}
\newcommand{\red}[1]{\vpha{\color{red}#1}}
\newcommand{\brw}[1]{\vpha{\color{brown}#1}}
\newcommand{\ppl}[1]{\vpha{\color{purple}#1}}
\newcommand{\vlt}[1]{\vpha{\color{violet}#1}}


\makeatletter
\definearrow1{s>}{%
\ifx\@empty#1\@empty
\expandafter\draw\expandafter[\CF@arrow@current@style,-CF](\CF@arrow@start@node)--(\CF@arrow@end@node);%
\else
\def\curvedarrow@style{shorten <=\CF@arrow@offset,shorten >=\CF@arrow@offset,}%
\CF@expadd@tocs\curvedarrow@style\CF@arrow@current@style
\expandafter\draw\expandafter[\curvedarrow@style,-CF](\CF@arrow@start@name)..controls#1..(\CF@arrow@end@name);
\fi
}
\makeatother



\usetikzlibrary{decorations}
\pgfdeclaredecoration{ddbond}{initial}{
  \state{initial}[width=2pt]{
    \pgfpathlineto{\pgfpoint{4pt}{0pt}}
    \pgfpathmoveto{\pgfpoint{1pt}{2pt}}
    \pgfpathlineto{\pgfpoint{2pt}{2pt}}
    \pgfpathmoveto{\pgfpoint{4pt}{0pt}}
  }
  \state{final}{
    \pgfpathlineto{\pgfpointdecoratedpathlast}
  }
}
\tikzset{lddbond/.style={decorate, decoration=ddbond}}
\tikzset{rddbond/.style={decorate, decoration={ddbond, mirror}}}


\begin{document}

\abovedisplayskip=9pt plus 3pt minus 9pt
\abovedisplayshortskip=0pt plus 3pt
\belowdisplayskip=9pt plus 3pt minus 9pt
\belowdisplayshortskip=4pt plus 3pt minus 4pt

\definesubmol\R{ \chemskipalign{\mbox{\scriptsize R}} }
\definesubmol\no{\vpha-[,0.1,,,draw=none]}
\definesubmol\Me[H_3C]{CH_3}
\definesubmol\Ammonia{\Lewis{4:,N}H_3}
\definesubmol\Cyanide{\quad\llap{${}^-$}\Lewis{4:,C}~\Lewis{0:,N}}
\definesubmol\Hydroxide{\quad\llap{${}^-$}OH}
\definesubmol\Water{H_2O}
\definesubmol\Alkene{C([:120]-)([:-120]-)=C([:-60]-)([:60]-)}
\definesubmol\O{\red{O}}
\definesubmol\N{\blu{N}}
\definesubmol\S{\brw{S}}
\definesubmol\X{H}
\newcommand{\M}{\chemabove{\N}{\X}}
\newcommand{\K}{\chembelow{\N}{\X}}

%\setatomsep{2em}
\setchemfig{atom sep=2.2em}
%\setcrambond{2pt}{0.75pt}{0.75pt}
\setchemfig{cram width = 4pt}
\setchemfig{cram dash width = 0.75pt}
\setchemfig{cram dash sep = 0.75pt}
\setchemfig{arrow label sep = 1.5pt}
\setchemfig{bond join=true}
%\setchemfig{atom style={scale=1.6}}
%\setchemfig{bond style={thick}}


\begin{preview}

\quad\chemname{\chemfig{*6(----\O--)}}{oxolane}
\quad\chemname{\chemfig{*6(----\M--)}}{piperidine}
\quad\chemname{\chemfig{*6(----\S--)}}{thiolane}
\\ \\ \\

\quad\chemname{\chemfig{*6(-=--\O-=)}}{pyran}
\quad\chemname{\chemfig{*6(-=-=\N-=)}}{pyridine}
\quad\chemname{\chemfig{*6(-=--\S-=)}}{thiopyran}
\\ \\ \\

\quad\chemname{\chemfig{*6(-\K---\O--)}}{\\morpholine}
\quad\chemname{\chemfig{*6(-\K---\M--)}}{\\piperazine}
\quad\chemname{\chemfig{*6(-\K---\S--)}}{\\thiazinane}
\\ \\ \\

\quad\chemname{\chemfig{*6(-\K-=-\O-=)}}{\\oxazine}
\quad\chemname{\chemfig{*6(-\N=-=\N-=)}}{\\pyrazine}
\quad\chemname{\chemfig{*6(-\K-=-\S-=)}}{\\thiazine}
\\ \\ \\

\quad\chemname{\chemfig{*6(-\O---\O--)}}{dioxane}
\quad\chemname{\chemfig{*6(-=\N-=\N-=)}}{pyrimidine}
\quad\chemname{\chemfig{*6(-\S---\S--)}}{dioxane}
\\ \\ \\

\quad\chemname{\chemfig{*6(-\O-=-\O-=)}}{dioxine}
\quad\chemname{\chemfig{*6(-=-\N=\N-=)}}{pyridazine}
\\ \\ \\

\quad\chemname{\chemfig{\O*6(--\O--\O--)}}{trioxane}
\quad\chemname{\chemfig{\N*6(-=\N-=\N-=)}}{triazine}
\quad\chemname{\chemfig{\N*6(-\N=-\N=\N-=)}}{tetrazine}
\\ \\ \\

\quad\chemname{\chemfig{[:38.6]*7(----\O---)}}{oxepane}
\quad\chemname{\chemfig{[:38.6]*7(----\M---)}}{azepane}
\quad\chemname{\chemfig{[:38.6]*7(----\S---)}}{thiepane}
\\ \\ \\

\quad\chemname{\chemfig{*6(-=(*5(-=--))-=-=)}}{indene}
\chemname{
  \chemfig{*6(-(-[::150,1.15]?)---?[]--)}
  \chemfig{<[:15](>[:85,1.8]?)>[:-15]-[:60]-[:165]?-[:195]-[:240]}
} {norbornane}
\\ \\ \\

\quad\chemname{\chemfig{*6(-=(*5(-=-\O-))-=-=)}}{benzofuran}
\quad\chemname{\chemfig{*6(-=(*5(-=-\M-))-=-=)}}{indole}
\quad\chemname{\chemfig{*6(-=(*5(-=-\S-))-=-=)}}{benzothiophene}
\\ \\ \\

\quad\chemname{\chemfig{*6(-=(*5(-\N=-\O-))-=-=)}}{benzoxazole}
\quad\chemname{\chemfig{*6(-=(*5(-\N=-\M-))-=-=)}}{benzimidazole}
\quad\chemname{\chemfig{*6(-=(*5(-\N=-\S-))-=-=)}}{benzothiazole}
\\ \\ \\

\quad\chemname{\chemfig{\N*6(-=(*5(-\N=-\M-))-=\N-=)}}{purine}
\\ \\ \\

\quad\chemname{\chemfig{*6(-=(*6(-=-=-))-=-=)}}{naphthalene}
\chemname{
  \chemfig{*6(-(-[::90,0.58]-[::60,0.58]?)--(-[::90,0.58]?)--(-[::90,0.58]?)-)}
  \chemfig{-[:15](-[:90]>[:60]?)-[:-15]>[:60](-[:90]?)
           -[:195]-[:165]()(-[:90]?)<[:240]}
} {adamantane}
\\ \\ \\





\end{preview}

\end{document}

