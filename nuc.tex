\documentclass[leqno,a4paper,twoside,aps,pra,preprint,10pt,notitlepage]{revtex4-2}
\usepackage[active,tightpage]{preview}
\usepackage[margin=0mm,paperwidth=480pt]{geometry}
\usepackage{parskip}
\usepackage[utf8]{inputenc}
\usepackage{environ}
\usepackage{xargs}
\usepackage{chemfig}
\usetikzlibrary{decorations.pathmorphing}
\usepackage{longtable}
\usepackage{amssymb}
\usepackage{multirow}
\usepackage[verbose]{wrapfig}
\usepackage{textpos}
\usepackage{makecell}
\usepackage{array}
\usepackage{amsmath}
\usepackage{ifthen}
\usepackage{cleveref}
\usepackage{color}
\usepackage{microtype}


\pagestyle{empty}
\setlength{\parindent}{0em}
\DisableLigatures{encoding = *, family = *}

\counterwithin*{equation}{section}
\renewcommand{\baselinestretch}{1.0}
\renewcommand{\theequation}{{{\bf{{\color{red}\arabic{section}.\arabic{equation}}}}}}
\renewcommand{\thesection}{}
\renewcommand{\thesubsection}{}
\renewcommand{\thesubsubsection}{}
\makeatletter
\def\@seccntformat#1{\csname #1ignore\expandafter\endcsname\csname the#1\endcsname{}}
\let\sectionignore\@gobbletwo
\let\latex@numberline\numberline
\def\numberline#1{\if\relax#1\relax\else\latex@numberline{#1}\fi}
\makeatother

\makeatletter
\let\oldtagform@\tagform@
\renewcommand\tagform@[1]{\maketag@@@{\ignorespaces#1\unskip\@@italiccorr}}
\renewcommand{\eqref}[1]{\textup{\oldtagform@{\ref{#1}}}}
\makeatother


\newcommand{\vpha}{\vphantom{A}}
\newcommand{\degree}{\ensuremath{^\circ}}
\newcommand{\chemar}[2][]{
  \schemestart[{},#1]
  \vpha\arrow{#2}\vpha
  \schemestop
}
\newcommand{\chemeq}[3][]{
  \chemnameinit{}
  \noindent
  \begin{equation}
    \schemestart[][base]
    #3
    \schemestop
    %\hskip \textwidth minus \textwidth
    \ifthenelse{\equal{#1}{}}{}{\hskip 10mm\rm{also\ }\cref{#1}}
    \ifthenelse{\equal{#2}{}}{}{\label{#2}}
  \end{equation}
  \vskip 1ex
}
\newcommand{\precyc}{\vpha-[,0.1,,,draw=none]}
\newcommand{\grn}[1]{\vpha{\color{green}#1}}
\newcommand{\blu}[1]{\vpha{\color{blue}#1}}
\newcommand{\blk}[1]{\vpha{\color{black}#1}}
\newcommand{\red}[1]{\vpha{\color{red}#1}}
\newcommand{\brw}[1]{\vpha{\color{brown}#1}}
\newcommand{\ppl}[1]{\vpha{\color{purple}#1}}
\newcommand{\vlt}[1]{\vpha{\color{violet}#1}}
\definesubmol\O{\red{O}}
\definesubmol\N{\blu{N}}
\definesubmol\S{\brw{S}}

\definesubmol\no{\vpha-[,0.1,,,draw=none]}
\definesubmol\Me[H_3C]{CH_3}
\definesubmol\Ammonia{\Lewis{4:,\N}H_3}
\definesubmol\Cyanide{\quad\llap{${}^-$}\Lewis{4:,C}~\Lewis{0:,\N}}
\definesubmol\Hydroxide{\quad\llap{${}^-$}\O|H}
\definesubmol\Water{H_2\O}
\definesubmol\Alkene{C([:120]-)([:-120]-)=C([:-60]-)([:60]-)}
\definesubmol\R{ \chemskipalign{\mbox{\scriptsize R}} }
\definesubmol{x}{(<[4]H)(<[0]\O|H)}
\definesubmol{y}{(<[0]H)(<[4]H|\O)}
\definesubmol{z}{CH\O|H}
\definesubmol{Z}{{(}CH\O|H{)}_n}



\makeatletter
\definearrow1{s>}{%
\ifx\@empty#1\@empty
\expandafter\draw\expandafter[\CF@arrow@current@style,-CF](\CF@arrow@start@node)--(\CF@arrow@end@node);%
\else
\def\curvedarrow@style{shorten <=\CF@arrow@offset,shorten >=\CF@arrow@offset,}%
\CF@expadd@tocs\curvedarrow@style\CF@arrow@current@style
\expandafter\draw\expandafter[\curvedarrow@style,-CF](\CF@arrow@start@name)..controls#1..(\CF@arrow@end@name);
\fi
}
\makeatother



\let\origsection\section
\makeatletter
\renewcommand{\section}[1]{%

    \vphantom{.}

    \vspace{-5.5ex}
    \color[rgb]{0.36, 0.54, 0.66}
    \origsection{\large #1}
    \color[gray]{0}
    \vspace{-3ex}
}
\makeatother

\let\origsubsection\subsection
\makeatletter
\renewcommand{\subsection}[1]{%

    \vphantom{.}

    \vspace{-6ex}
    \origsubsection{\large #1}
    \vspace{-3ex}
}
\makeatother

\let\origsubsubsection\subsubsection
\makeatletter
\renewcommand{\subsubsection}[1]{%

    \vphantom{.}

    \vspace{-6.5ex}
    \origsubsubsection{\large #1}
    \vspace{-3ex}
}
\makeatother






\begin{document}

\abovedisplayskip=0ex
\abovedisplayshortskip=0pt
\belowdisplayskip=0ex
\belowdisplayshortskip=0ex

\setchemfig{atom sep=2em}
%\setcrambond{2pt}{0.75pt}{0.75pt}
\setchemfig{cram width = 2pt}
\setchemfig{cram dash width = 0.75pt}
\setchemfig{cram dash sep = 0.75pt}
\setchemfig{arrow label sep = 1.5pt}
\setchemfig{bond join=true}








\begin{preview}


\definesubmol{a}{-P(=[::-90,0.75,,1]\O)(-[::90,0.75,,1]\O|H)-}

\definesubmol{tail}{
    *5(-\O-(--[::60]\O([::-60]!aH\O))<(-[,,,1]\O|H)
    -[,,,,line width=2pt](-\O|H)>)
}

\definesubmol{sail}{
    *5(<(-H|\O)
    -[,,,,line width=2pt](-[,,,1]\O|H)>(--[::60]\O([::-60]!a\O|H))-\O-)
}

\chemfig{[:18]*5((-H|\O)-[,,,,line width=2pt](-[,,,1]\O|H)>(-)-\O-(--[::60]\O([::-60]!aH|\O))<)}
\\
\\
\\


\chemfig{\N*5(-\O----)}


\definesubmol{guanine}{
  @{hb-gua1}H-[:180]\N(-[:-120]H)
  -[:120]*6(
    -\N(-@{hb-gua2}H)
    -(=@{hb-gua3}\O)
    -(*5(-\N=-\N(-!{tail})-))
    =-\N=
  )
}

\definesubmol{cytosine}{
  @{hb-cyt1}\O=[:60]*6(
    -\N(-!{sail})
    -=-(
      -\N(-[::60]@{hb-cyt3}H)
      -[::-60]H
    )
    =@{hb-cyt2}\N-
  )
}


\chemname{\chemfig{!{guanine}}}{Guanine}
\qquad
\chemname{\chemfig{!{cytosine}}}{Cytosine}
\chemmove[-,dashed]{\foreach \i in {1,2,3} { \draw (hb-gua\i) -- (hb-cyt\i) ;}}



\definesubmol{adenine}{
  @{hb-ade1}H-[:180]\N(-[:120]H)-[:-120]
  *6(=(*5(-\N=-\N(-!{tail})-))-=\N-=@{hb-ade2}\N-)
}

\definesubmol{uracil}{
  @{hb-ura1}\O=[:-60]*6(
    -\N(-[:180]@{hb-ura2}H)-(=\O)-\N(-!{sail})
    -=(-)-
    =
  )
}


\chemname{\chemfig{!{adenine}}}{Adenine}
\qquad
\chemname{\chemfig{!{uracil}}}{Uracil (RNA) Thymine (DNA)}
\chemmove[-,dashed]{\foreach \i in {1,2} { \draw (hb-ade\i) -- (hb-ura\i) ;}}






\end{preview}

\end{document}
